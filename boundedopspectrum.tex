\section{Bounded operator spectrum}
\noindent
All spaces in out theory are considered to be Banach spaces, we also operate on
complex numbers, since it is easier to reason about.

\begin{defn}
  $A\colon X \to X$ --- linear bounded operator or $A \in L(X)$,
  $\lambda \in \C,\ Ix = x$.
  If operator $A - \lambda I$ is continuously reversible, then
  $R_\lambda(A) = (A - \lambda I)^{-1}$ is called \textbf{resolvent operator} and
  $\lambda$ is a \textbf{regular point} of $A$. 
\end{defn}

\begin{defn}
  $\rho(A)$ --- set of all regular
  points is called \textbf{resolvent set}. 
\end{defn}

\begin{defn}
  $\sigma(A) = \C \setminus \rho(A)$ is called \textbf{spectrum}
  of $A$.
\end{defn}

\begin{thm}[About bounded operator spectrum]
  $\forall A \in L(X) \implies \sigma(A) \neq \varnothing$
\end{thm}

\begin{thm}
  $\rho(A)$ is open in $\C$.
\end{thm}

\begin{proof}
  \[
    \forall \lambda_0 \in \rho(A)\ \exists \delta > 0:\ |\lambda - \lambda_0| <
    \delta \overset{?}{\implies} \lambda \in \rho(A)
  \]
  By Banach theorem:
  \begin{align*}
    &C \in L(X),\ \|C\| < 1 \implies I - C \text{ --- continuously reversible},\ (I -
      C)^{-1} = \sum_{n = 0}^\infty C^n\\
    &A - \lambda I = (A - \lambda_0 I) - (\lambda - \lambda_0) I \\
    &I = (A - \lambda_0 I)R_{\lambda_0}(A)\\
    &A -\lambda I = \underbrace{(A - \lambda_0 I)}_{\text{cont. reversible}}(I - (\lambda - \lambda_0)R_{\lambda_0}(A)) \\
    &|\lambda - \lambda_0| \cdot \|R_{\lambda_0}(A)\| < 1\\
    &|\lambda - \lambda_0| < \underbrace{\frac{1}{\|R_{\lambda_0}(A)\|}}_\delta \implies \lambda \in \rho(A) \qedhere
  \end{align*}
\end{proof}

\begin{thm}
  $\sigma(A) \subset \{\lambda : \abs{\lambda} \leq \|A\| \}$
\end{thm}

\begin{proof}
  \begin{align*}
    &|\lambda| > \|A\| \implies \lambda \in \rho(A)\\
    &A - \lambda I = -\lambda \underbrace{(I - \tfrac 1\lambda A)}_{\text{cont. reversible}} \\
    &\|\tfrac 1 \lambda A\| < 1 \quad \lambda \in \rho(A) \qedhere
  \end{align*}
\end{proof}

\begin{defn}
  $r_\sigma(A) = r_\sigma = \inf\limits_{n \in \N} \sqrt[n]{\|A^n\|}$
\end{defn}

\begin{thm}\leavevmode
  \begin{enumerate}
  \item $r_\sigma(A) = \lim\limits_{n \to \infty} \sqrt[n]{\|A^n\|}$ 
  \item $\sigma(A) \subset \Set{\lambda : \abs{\lambda} \leq r_\sigma(A)}$
  \end{enumerate}
\end{thm}

\begin{proof}\leavevmode
  \begin{enumerate}
  \item 
    $\begin{aligned}[t]
      &\forall \epsilon > 0\ \exists n_0 : \|A^n\|^{\frac{1}{n_0}} < r_\sigma + \epsilon \\
      &\forall n > n_0\quad  n = p_n \cdot n + d_n,\ d_n = 0,1,\dotsc, n_0 - 1 \\
      &\|A^n\|^{\frac 1n} = \|A^{p_n \cdot n + d_n }\|^{\frac{1}{p_n \cdot n + d_n}} \leq {(\|A^{n_0}\|^{p_n} \cdot \|A\|^{d_n})}^{\frac{1}{p_n \cdot n + d_n}} = (\|A^{n_0}\|^{\frac{1}{n_0}})^{\frac{p_n \cdot n_0}{n}} \cdot \tendsto{\|A\|^{\frac{d_n}{n}}}{1} \\
      & r_\sigma \leq \tendsto{\|A^n\|^{\frac{1}{n}}}{r_\sigma} \leq r_\sigma + \epsilon\ \forall n > n_1 > n_0
    \end{aligned}$
  \item
    $\begin{aligned}[t]
      &A - \lambda I = -\lambda (I - \frac 1\lambda A) \\
      &\sum_{n = 0}^\infty \frac{1}{\lambda^n} A^n,\ \sum_{n = 0}^\infty \|A^n\| \frac{1}{|\lambda|^n},\ t = \frac{1}{|\lambda|},\ R = \frac{1}{\lim \sqrt[n]{\|A^n\|}}\\
      &|\lambda| > \lim \sqrt[n]{\|A^n\|} = r_\sigma && \qedhere
    \end{aligned}$ 
  \end{enumerate} 
\end{proof}

\begin{note}[Inaccurate spectrum estimations]
  \begin{align*}
    &l_2,\ (x_1, x_2, \dots) \mapsto (0, x_1, x_2, \dots) \\
    &A \colon l_2 \to l_2\ \|Ax\| = \|x\|,\ \|A^n x\| = \|x\| \implies \|A^n\| = 1 \implies r_\sigma = 1
  \end{align*}
  Now let's find $A$ spectrum using previous formula. It's easy to see that
  $\lambda = 0 \in \sigma(A)$. Consider $\lambda \neq 0$:
  \begin{align*}
    &(A - \lambda I) x = (0, x_1, x_2 \dots) - (\lambda x_1, \lambda x_2, \dots) = (-\lambda x_1, x_1 - \lambda x_2, x_2 - \lambda x_3, \dots) = (y_1, y_2, \dots) \in l_2 \\
    & \begin{cases}
      -\lambda x_1 = y_1,\quad x_1 = \frac{y_1}{-\lambda}\\
      x_1 - \lambda x_2 = y_2,\quad x_2 = \frac{y_2 - x_1}{-\lambda}\\
      x_2 - \lambda x_3 = y_3,\quad x_3 = \frac{y_3 - x_2}{-\lambda}\\
      \vdots
      \end{cases}
  \end{align*}
    Hence, $\forall \lambda \neq 0 \quad \lambda \in \rho(A)$. Thus,
    $\sigma(A) = \{0\},\ r_\sigma = 1$
\end{note}

To prove the main theorem we will need so called analytical operator-functions.
\begin{defn}
  Let $A_n \in L(X)$, then 
   $\sum\limits_{n = 0}^\infty A_n \lambda^n,\ \lambda \in \C$ is called
   \textbf{analytical operator-function}.
\end{defn}

\begin{note}
  Because $X$ is a Banach space, formal theory of such series is the same as for
  complex numerical series, which means we can discuss things like convergence radius.
  \[
    R = \frac{1}{\varlimsup \sqrt[n]{\|A_k\|}}
  \]
\end{note}

\begin{lemma}[Abel]
  \[
    \sum_{n = 0}^\infty A_n \lambda_0^n,\ |\lambda| < |\lambda_0| \implies
    \sum_{n = 0}^\infty \|A_n\| |\lambda|^n < +\infty
  \]
\end{lemma}

\begin{proof}\leavevmode
    \[
      \|A^n\| |\lambda|^n = \tendsto{\|A_n \cdot \lambda_0^n\|}{0} \cdot
      \biggl( \frac{|\lambda|}{|\lambda_0|} \biggl)^n < \underbrace{\biggl(
      \frac{|\lambda|}{|\lambda_0|} \biggl)^n}_{\text{converges}} \qedhere
    \]
\end{proof}

\begin{thm}[Liouville]
  \[
    f(\lambda) = \sum_{n = 0}^\infty a_n \lambda,\ 
    |f(\lambda)| \leq M \implies f \equiv const
  \]
\end{thm}

\begin{proof}
    %R_\lambda(A) = (A - \lambda I)^{-1},\ \lambda \in \rho(A) \\
  \begin{gather*}
    \begin{split}
      A - \lambda I & = (A - \lambda_0 I) - (\lambda - \lambda_0) I = (A - \lambda_0 I) - (\lambda - \lambda_0) (A - \lambda_0 I) R_{\lambda_0}(A) = {}\\
                    & = (A - \lambda_0 I)(I - (\lambda - \lambda_0)R_{\lambda_0}(A)),\ \lambda_0 \in \rho(A) 
    \end{split} \\
    (I - (\lambda - \lambda_0) R_{\lambda_0}(A))^{-1} = \sum_{n=0}^\infty R_{\lambda_0}^n(A)(\lambda - \lambda_0)^n \\
    R_\lambda(A)  = \sum_{n=0}^\infty R_{\lambda_0}^{n + 1} (\lambda - \lambda_0)^n,\ \lambda \sim \lambda_0 \\
    A - \lambda I = -\lambda(I - \tfrac 1\lambda A),\ \lambda \sim \infty \\
    R_\lambda(A)  = -\lambda \sum_{n = 0}^\infty A^n \frac{1}{\lambda^n}
  \end{gather*}
  $R_\lambda$ is analytical operator-function in infinitely remote point.
  Let's assume that $\sigma(A) = \varnothing$. 
  $R_\lambda$ is analytical in all complex numbers, but it is bounded on all
  complex number. Then by Liouville theorem $R_\lambda(A) \equiv const \implies
  f \equiv const \contr$
\end{proof}
