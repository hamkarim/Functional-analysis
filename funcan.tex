\documentclass[12pt, fleqn]{article}
\usepackage{titlesec}
\usepackage{amssymb}
\usepackage{amsmath}
\usepackage{amsthm}
\usepackage{mathtools}

\theoremstyle{definition}
\newtheorem*{defn}{Definition} 
\newtheorem{ex}{Example}

\DeclarePairedDelimiter\abs{\lvert}{\rvert}%
\DeclarePairedDelimiter\norm{\lVert}{\rVert}%

\makeatletter
\let\oldabs\abs
\def\abs{\@ifstar{\oldabs}{\oldabs*}}

\let\oldnorm\norm
\def\norm{\@ifstar{\oldnorm}{\oldnorm*}}
\makeatother

\newtheoremstyle{break}% name
  {}%         Space above, empty = `usual value'
  {}%         Space below
  {\itshape}% Body font
  {}%         Indent amount (empty = no indent, \parindent = para indent)
  {\bfseries}% Thm head font
  {.}%        Punctuation after thm head
  {\newline}% Space after thm head: \newline = linebreak
  {}%         Thm head spec
\theoremstyle{break}
\newtheorem{thm}{Theorem}[section]

\theoremstyle{theorem}
\newtheorem{stm}{Statement}[section]
\newtheorem{cor}{Corollary}[thm]
\newtheorem{lemma}{Lemma}

\renewcommand\qedsymbol{$\blacksquare$}

\newcommand{\verteq}{\rotatebox{90}{$\,=$}}
\newcommand{\equalto}[2]{\underset{\scriptstyle\overset{\mkern4mu\verteq}{#2}}{#1}}
\newcommand{\defeq}{\overset{def}{=}}
\newcommand{\tendsto}[2]{\underbrace{#1}_{\underset{#2}{\downarrow}}}
\newcommand{\RR}{\mathbb{R}}
\newcommand{\ZZ}{\mathbb{ZZ}}
\newcommand{\NN}{\mathbb{N}}

\DeclareMathOperator{\clOp}{Cl}
\newcommand{\cl}[1]{\clOp({#1})}
\DeclareMathOperator{\intOp}{Int}
\newcommand{\inter}[1]{\intOp({#1})}
\DeclareMathOperator{\frOp}{Fr}
\newcommand{\fr}[1]{\frOp({#1})}
\DeclareMathOperator{\diamOp}{diam}
\newcommand{\diam}[1]{\diamOp({#1})}

\titleformat{\subsection}
  {\centering\normalfont\Large}{\S\thesubsection}{1em}{}
\titleformat{\section}
  {\centering\normalfont\Large\bfseries}{\thesection}{1em}{}
  
\author{Sugak A.M.}
\title{Functional analysis course by Dodonov N.U.}
\date{Fall 2015 --- Spring 2016} 

\begin{document}
\maketitle
\newpage
\section{Vector spaces.}
\subsection{Metric spaces.}

X, $\rho\colon X \times X \to \mathbb{R}_+$
\begin{defn}$\rho$ --- \textbf{metric}\end{defn}
  \begin{enumerate} 
    \item $\rho(x, y) \geq 0, = 0 \iff x = y$
    \item $\rho(x, y) = \rho (y, x)$
    \item $\rho(x, y) \leq \rho (x, z) + \rho (y, z)$
  \end{enumerate}
\begin{defn}$(X, \rho)$ --- \textbf{Metric space}.\end{defn}
\begin{defn}$x = \lim x_{n} \iff \rho(x_{n}, x) \rightarrow 0$\end{defn}
\vspace{5mm}
$X, \tau = \{G \subset X\}$
\begin{enumerate}
  \item $\varnothing, X \in \tau$
  \item $G_\alpha \in \tau, \alpha \in \mathcal{A} \implies \bigcup\limits_\alpha G_\alpha \in \tau$
  \item $G_1, \dotsc, G_n \in \tau \implies \bigcap\limits_{j = 1}^n G_j \in \tau$
\end{enumerate}
\begin{defn}$(X, \tau)$ --- \textbf{Topological space}.\end{defn}
$x = \lim x_n \quad \forall G \in \tau: x \in G \quad \exists N: \forall n > N \implies x_n \in G$
\\G --- open in $\tau$
\\$F = X \setminus G$ --- closed
\begin{defn}
  $B_r (a) = \{x: \rho(x, a) < r\}$ --- \textbf{open ball}
\end{defn}
\noindent
$\tau = \bigcup B_r (a)$
\vspace{5mm}
\begin{stm}
  $b \in B_{r_1} (a_1) \cap B_{r_2} (a_2) \implies \exists r_3 > 0: B_{r_3}
  (a_3) \subset B_{r_1} (a_1) \cap B_{r_2} (a_2)$
\end{stm}
\noindent
In this sense metric space is just a special case of topological space.
\begin{ex}
  $\mathbb{R}, \rho(x, y) = |x - y|$, MS
\end{ex}
\begin{ex}
  $\bar{x} = (x_1, \dotsc, x_n) \in \mathbb{R}^n, \rho(\bar{x}, \bar{y}) = \sqrt{\sum\limits_{j = 1}^n(x_j - y_j)^2}, MS$
\end{ex}
\begin{ex}
  $\bar{x} = (x_1, \dotsc, x_n, \dotso) \in \mathbb{R}^\infty
  \\\alpha\bar{x} = (\alpha x_1, \dotsc, \alpha x_n, \dotso)
  \\\bar{x} + \bar{y} = (x_1 + y_1, \dotsc, x_n + y_n, \dotso)\\\lim\limits_{m \to \infty}\bar{x}_m?
  \\\textnormal{in}\ \mathbb{R}^n\ \bar{x}_n \to \bar{x} \iff \forall j = 1 \dotso n\quad x_j^{(m)} \underset{m \to \infty}{\to} x_j
  \\\textnormal{in}\ \mathbb{R}^{\infty}\ \bar{x}_m \to \bar{x} \overset{def}{\iff} \forall j = 1,2,3,\dotso \quad x_j^{(m)} \underset{m \to \infty}{\to} x_j$
  \begin{defn}
    $\rho(\bar{x}, \bar{y}) \defeq \sum\limits_{n = 1}^\infty \frac{1}{2^n}\underbrace{\frac{|x_n - y_n|}{1 + |x_n - y_N|}}_{\phi(|x_n - y_n|)}$ 
    --- \textbf{Urysohn metric}.
  \end{defn}
  $\\\phi(t) = \frac{t}{1 + t}
  \\\phi(t_1 + t_2) \leq \phi(t_1) + \phi(t_2)
  \\\rho(\bar{x_m}, \bar{x}) \underset{m \to \infty}{\to} 0 \iff x_j^{(m)} \to x_j\ \forall j$
  \\In this way $\mathbb{R}^\infty$ is a metrizable space.
\end{ex}
\begin{ex}
$X, \rho(x, y) \defeq
\begin{cases}
  0, x = y
  \\1, x \neq y
\end{cases}$ --- \textbf{Discrete metric}.
$\\x_n \to x\ \mathcal{E} = \frac{1}{2}\ \exists M:\ m > M \implies \rho(x_m, x) < \frac{1}{2} \implies 
\\\rho(x_m, x) = 0 \implies x_m = x$
\end{ex}

\begin{defn}
  $(X, \tau);\ \forall A \subset X;\ 
  \\\inter{A} \defeq \bigcup\limits_{G \subset A} G - \textnormal{open}; 
  \\\cl{A} \defeq \bigcap\limits_{A \subset G} G - \textnormal{closed};
  \\\fr{A} = \cl{A} \setminus \inter{A}$
\end{defn}
\vspace{5mm}
\noindent
$(X, \rho);$ Having a metric space one can describe closure of a set.
$\\\rho(x, A) \defeq \inf\limits_{a \in A}\rho(x, a)
\\\rho(A, B) = \inf\limits_{\substack{a \in A\\b \in B}}\rho(a, b)
\\\rho(x, A) = f(x), x \in X$
\begin{stm}
  Function f(x) is continuous.
\end{stm}
\begin{proof}
  $\forall x, y \in X
  \\f(x) = \rho(x, A) \underset{\forall \alpha \in A}{\leq} \rho(x, \alpha) \leq \rho(x, y) + \rho(y, \alpha)
  \\\forall \mathcal{E} > 0\ \exists \alpha_\epsilon \in A:\ \rho(y, \alpha_\epsilon) < \rho(y, A) + \mathcal{E} = f(y) + \mathcal{E}
  \\f(x) \leq f(y) + \mathcal{E} + \rho(x, y),\ \mathcal{E} \to 0\\
  \begin{cases}
    f(x) \leq f(y) + \rho(x, y)
    \\f(y) \leq f(x) + \rho(x, y)
  \end{cases} \implies |f(x) - f(y)| \leq \rho(x, y)\qedhere$
\end{proof}
\begin{stm}
  $x \in \cl{A} \iff \rho(x, A) = 0$
\end{stm}
Let's look at the metric spaces in terms of separation of sets from each other by open sets.
$\\x, y
\\r = \rho(x, y) > 0
\\B_{\frac{r}{3}}(x),\ B_{\frac{r}{3}}(y)$
\\In any metric space separability axiom is true.
\begin{thm}
  Any metric space is a normal space, \\i.e.
  $\forall\ closed\ disjoint\ F_1, F_2 \in X,\ \exists\ open\ disjoint\ G_1, G_2\colon F_j \in G_j,\ j = 1, 2$
\end{thm}
\begin{proof}
  $g(x) = \frac{\rho(x, F_1)}{\rho(x, F_1) + \rho(x, F_2)}$ - continuous on X
  $\\x \in F_1,\ \cl{F_1} = F_1,\ \rho(x, F_1) = 0,\ g(x) = 0
  \\x \in F_2,\ g(x) = 1$
  \\Let's look at $(-\infty; \frac{1}{3}), (\frac{2}{3}, \infty)$ --- by continuity their inverse images under g are open.
  $\\G_1 = g^{-1}(-\infty; \frac{1}{3})
  \\G_2 = g^{-1}(\frac{2}{3}; \infty)\qedhere$
\end{proof}
\begin{defn}
  Metric space is \textbf{complete} if $\rho(x_n, x_m) \to 0 \implies \exists x = \lim x_n
  \\\mathbb{R}^\infty$ -- complete (by completeness of the rational numbers).
  \\In complete metric spaces the nested balls principle is true.
\end{defn}
\begin{thm}
  X -- complete metric space, $\overline{V}_{r_n}$ -- system of closed balls.
  \\\begin{enumerate}
      \item $\overline{V}_{r_{n + 1}} \subset \overline{V}_{r_n}$ -- the system is nested.
      \item $r_n \to 0$
    \end{enumerate}
  \underline{Then:} $\bigcap\limits_n \overline{V}_{r_n} = \{a\}$
\end{thm}
\begin{proof}
  Let $b_n$ be centers of $\overline{V}_{r_n},
  \\m \geq n,\ b_m \in \overline{V}_{r_n},\ \rho(b_m, b_n) \leq r_n \to 0\ \forall m \geq n
  \\\rho(b_m, b_n) \to 0 \overset{comp.}{\implies} \exists a = \lim b_n$
  Since the balls are closed a $\in$ every ball.
  $\\r_n \to 0 \implies$ there is only one common point $\qedhere$.
\end{proof}
\noindent
$(X, \tau)$ --- topological space
$\\A \subset X,\ \tau_a = \{G \cap A, G \in \tau\}$
\begin{defn}
  $X \textnormal{--- metric space},\ A \subset X,\ \cl{A} = X$
  \\\underline{Then:} A -- \textbf{dense} in X
  \\\underline{If} $\inter{\cl{A}} = \varnothing$ A -- \textbf{nowhere dense} in X.
  \\It is easy to understand, that in metric spaces nowhere density means the following: 
  $\forall\ ball\ V\ \exists V^{'} \subset V\colon V^{'}$ contains no points from A.
\end{defn}
X is called \textbf{first Baire category set}, if it can be written as at most countable union of $x_n$ each nowhere dense in X.
\begin{thm}[Baire category theorem]
  Complete metric space is second Baire category set in itself.
\end{thm}
\begin{proof}
  Let X be first Baire category set.
  \\$X = \bigcup\limits_n X_n \quad \forall \overline{V}\ X_1\ \textnormal{is nowhere dense}.
  \\\overline{V}_1 \subset \overline{V}\colon\ \overline{V}_1 \cap X_1\ = \varnothing
  \\X_2\ \textnormal{is nowhere dense}\ \overline{V}_2 \subset \overline{V}_1 \colon \overline{V}_2 \cap X_2 = \varnothing
  \\r_2 \leq \frac{r_1}{2}
  \\\vdots
  \\\{\overline{V}_n\},\ r_n \to 0,\ \bigcap\limits_n \overline{V}_n = \{a\},\ X = \bigcup X_n,\ \exists n_0 \colon a \in X_{n_0}
  \\X_{n_0} \cap \overline{V}_{n_0} = \varnothing \to\leftarrow\ a \in \overline{V}_{n_0} \qedhere$
\end{proof}
\begin{cor}
  Complete metric space without isolated points is uncountable.
\end{cor}
\begin{proof}
  No isolated points are present $\implies$ every point in the set is nowhere dense in it. Let X be countable:
    $X = \bigcup\limits_n \{X_n\}$, then it is first Baire category set. $\to\leftarrow\qedhere$
\end{proof}
\begin{defn}
  K --- \textbf{compact} if
  \begin{enumerate}
    \item $K = \cl{K}$
    \item \label{itm:second}$x_n \in K\ \exists n_1 < n_2 < \dotso\ x_{n_j} - \textnormal{converges in X}$.
  \end{enumerate}
  If only \ref{itm:second} is present, the set is called \textbf{precompact}.
\end{defn}
\begin{thm}[Hausdorff]
  Let X --- metric space, K --- closed in X.
  \\\underline{Then:} K --- compact $\iff$ K --- totally bounded,
  \\i.e. $\forall \mathcal{E} > 0\ \exists a_1,\dotsc, a_p \in X \colon\ \forall b \in K\ \exists a_j \colon\ \rho(a_j, b) < \mathcal{E}
  \\(a_1, \dotsc, a_p - finite\ \mathcal{E}-net)$
\end{thm}
\begin{proof}
  $\\\implies
  \\K \textnormal{--- totally bounded},\ x_n \in K\ n_1 < n_2 < \dotso < n_k < \dotso
  \\x_n - \textnormal{converges in K}
  \\\mathcal{E}_k \downarrow \to 0\ \mathcal{E}_1 \quad K \subset \bigcup\limits_{j = 1}^p \overline{V}_j,\ rad = \mathcal{E}_1\ (\mathcal{E}_1 - net)
  \\\textnormal{n is finite }\implies \textnormal{one ball will contain infinetely many}\ x_n\ \textnormal{elements}.\ 
  \\\textnormal{Let's look at}\ \overline{V}_{j_0} \cap K\ \textnormal{--- totally bounded}\ = K_1,\ \diam{K_1} \leq 2\mathcal{E}_1
  \\\mathcal{E}_2\quad K_1 \subset \bigcup\limits_{j =1}^n
  \overline{V^{'}}_{j},\ rad = \mathcal{E}_2,\ \textnormal{then one of}\
  \overline{V^{'}}\ 
  \\\textnormal{contains infinitely many elements of the sequence contained in}\ K _1
  \\\overline{V^{'}}_{j_0} \cap K_1 = K_2,\ \diam{K_2} \leq 2\mathcal{E}_2\ \textnormal{and so on}.
  \\K_n \supset K_{n+1} \supset K_{n+2} \supset \dotso,\ \diam{K_N} \leq 2\mathcal{E}_n
  \overset{by\ space\ comp.}{\implies} \underbrace{\bigcap\limits_{n = 1}^\infty K_n}_{\diam{K_n} \to 0,\ \{x\}} \neq \varnothing$
  \\Take $x_{n_1}$ from $K_1$, $x_{n_2}$ from $K_2 \dotso$
  \\$\impliedby$
  \\K --- compact $\forall \mathcal{E}\ \exists$ finite $\mathcal{E}$\-net?
  \\By contradiction: $\exists\mathcal{E}_0 > 0\colon$\ finite $\mathcal{E}_0$-net is impossible to construct.
  \\$\forall x_1 \in K\ \exists x_2 \in K \colon\ \rho(x_1, x_2) > \mathcal{E}_0$ (or else system of $x_1$ --- finite $\mathcal{E}$-net)
  \\$\{x_1, x_2\}$ - choose $x_3 \in K\colon \rho(x_3, x_i) > \mathcal{E}_0,\ i = 1, 2$ and so on.
  \\$x_n \in K:\ n\ \neq m\ \rho(x_n, x_m) > \mathcal{E}_0$ --- contains no converging subsequence $\implies$
  set is not a compact. $\to\leftarrow\qedhere$
\end{proof}
\subsection{Normed spaces}
\begin{defn}
  X --- \textbf{linear set}, x + y, $\alpha \cdot x$, $\alpha \in \mathbb{R}$
  \\The purpose of norm definition, is to construct a topology on X, so that 2 linear operations are continuous on it.
\end{defn}
$\phi: X \to \mathbb{R}\colon$
\begin{enumerate}
\item $\phi(x) \geq 0,\ = 0 \iff x = 0$
\item $\phi(\alpha x) = |\alpha| \phi(x)$
\item $\phi(x + y) \leq \phi(x) + \phi(y)$
\end{enumerate}
\begin{defn}
  $\phi$ --- \textbf{norm} on X, $\phi(x) = \|x\|$
\end{defn}
$\rho(x, y) \defeq \|x - y\|$ --- metric on X. 
\begin{defn}
  $\\(X, \|\cdot\|)$ --- \textbf{normed space} --- special case of metrical space.
\end{defn}
$\\x = \lim x_n \overset{def}{\implies} \rho(x_n, x) \to 0 \iff \|x_n - x\| \to 0$
\begin{stm}
  In the topology of a normed space linear operations are continuous on X.
\end{stm}
\begin{proof}
  \begin{enumerate}
  \item 
    $\begin{aligned}[t]
      x_n \to x,\ y_n \to y;\ \|(x_n + y_n) - (x + y)\| & = \|(x_n - x) + (y_n - y)\|  \leq \\ 
      \tendsto{\|x_n - x\|}{0} + \tendsto{\|y_n - y\|}{0} & \implies x_n + y_n \to x + y
    \end{aligned}$
  \item 
    $\begin{aligned}[t]
       \alpha_n \to \alpha,\ x_n \to x;\ \|\alpha_n x_n - \alpha x\| =
        \|(\alpha_n - \alpha)x_n + \alpha(x_n - x)\| \leq \\
        \tendsto{|\alpha_n - \alpha|}{0} \cdot \underbrace{\|x_n\|}_{bounded} + \tendsto{\alpha\|x_n - x\|}{0}
    \end{aligned}$
    $\\x_n \to x \implies \|x_n\|$ --- bounded.
    $\\\alpha_x x_n \to \alpha x\qedhere$
  \end{enumerate}
\end{proof}
\begin{stm}
  From the triangle inequality $|\|x\| - \|y\|| \leq \|x - y\|
  \\x_n \to x \implies \|x_n\| \to \|x\|$
  \\Norm is continious.
\end{stm}
\begin{ex}
  $\mathbb{R}^n$ 
  \begin{enumerate}
  \item $\|\bar{x}\| = \sqrt{\sum\limits_{k = 1}^n x_k^2}$
  \item $\|\bar{x}\|_1 \defeq \sum\limits_{k = 1}^n|x_k|$
  \item $\|\bar{x}\|_2 \defeq \max \{|x_1| \dotso |x_n|\}$
  \item $C[a, b]$ --- functions continuous on $[a, b];\ \|f\| = \max\limits_{x \in [a, b]}|f(x)|$ 
  \item $L_p (E) = \{f - \textnormal{measurable},\ \int\limits_E |f|^p < +
    \infty\}\\p \geq 1,\ \|f\|_p = (\int\limits_E |f|^p)^{\frac 1p}$
  \end{enumerate}
\end{ex}
Because the set of points is the same, arises the question about convergence
comparison.
$\\\|\cdot\|_1 \sim \|\cdot\|_2,\ x_n \overset{\|\cdot\|_1}{\to} x \iff x_n \overset{\|\cdot\|_2}{\to}$
\begin{stm}
  $\\\|\cdot\|_1 \sim \|\cdot\|_2 \iff \exists a, b > 0\colon \forall x \in X
  \implies a\|x_1\|_1 \leq \|x\|_2 \leq b \|x\|_1$
\end{stm}
\begin{thm}[Riesz]
  $X,\ \dim{X} < +\infty$ --- linear set.
  \\\underline{Then:} Any pair of norms in X are equivalent.
\end{thm}
\begin{proof}
  $l_1, \dotsc, l_n$ --- linearly independent from X. $\forall x \in X =
  \sum\limits_{k = 1}^n \alpha_k l_K
  \\\bar{x} \leftrightarrow (l_1, \dotsc, l_n) = \bar{l} \in \mathbb{R}^n$
  \\Let $\|\cdot\|$ --- some norm in X.
  \\$\|x\| \underset{\triangle}{\leq} \sum\limits_{k = 1}^n \|l_k\| |\alpha_k|
  \underset{Cauchy}{\leq} \underbrace{\sqrt{\sum\limits_1^n \|l_k\|^2}}_{constB}
  \equalto{\sqrt{\sum\limits_1^n |\alpha_k|^2}}{\|\bar{\alpha{}}\| = \|x\|_1}
  \\\|x\|_1 = \sqrt{\sum\limits_1^n \|alpha_k\|^2},\ x = \sum \alpha_k l_k
  \\\|x\| \leq b \|x\|_1
  \\?\exists a > 0\colon a\|x\|_1 \leq \|x\| \implies \|\cdot\| \sim \|\cdot\|_1$
  \\Let $f(\alpha_1, \dotsc, \alpha_n) = \norm{\sum\limits_{k = 1}^n \alpha_k l_k}\\
  \begin{aligned}[t]
    \\|f(\bar{\alpha} + \Delta\bar{\alpha}) - f(\bar\alpha)  = &\abs{\norm{\sum\limits_1^k
    \alpha_k l_k + \sum\limits_1^n \Delta \alpha_k l_K } - \norm{\sum\limits_1^n
    \alpha_k l_k}} \leq \\ & \norm{\sum\limits_1^n \Delta \alpha_k l_k} \leq \tendsto{\sum
    \| l_k\| |\Delta \alpha_k|}{0, \Delta \alpha_k \to 0} \implies f
  \textnormal{--- continuous on}\ \RR{}^n
  \end{aligned}$
  $S_1 = \{\sum\limits_1^n \alpha_k|^2 = 1\} \subset \RR{}^m$, f --- continuous
  on $S_1$, $\bar{\alpha}^* \in S_1$
  \\$\forall \alpha \in S_1 \implies f(\bar{\alpha}^*) \leq f(\bar{\alpha})
  \\f(\bar{\alpha}^*) = 0
  \\\norm{\sum\limits_1^n \alpha_k^* l_k} = 0
  \\\sum\limits_1^n \alpha_k^* l_k = 0,\ \bar{\alpha}^* \in S_1
  \\l_1 \dots l_n\ \textnormal{--- linearly independent}\ \to\leftarrow
  \\\min\limits_{S_1} f = m > 0\\
  \begin{aligned}[t]
    \\\|x\| = \norm{\sum\limits_1^n \alpha_K l_K} & = f(\bar{\alpha}) =
    \sqrt{\sum\limits_1^n \alpha_k^2} \cdot \norm{\sum
    \underbrace{\frac{\alpha_k}{\sqrt{\sum\limits_1^n \alpha_k^2}}}_{\beta_k}
    l_k},\ \bar{\beta} = (\beta_1 \dots \beta_n) \in S_1 \\
    & \geq m \cdot \|x\|_1,\ a = m \qedhere 
  \end{aligned}$
\end{proof}
\end{document}
