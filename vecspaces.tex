\chapter{Vector spaces}
\section{Metric spaces.}

X, $\rho\colon X \times X \to \R_+$
\begin{defn}$\rho$ --- \textbf{metric}
  \begin{enumerate}
    \item $\rho(x, y) \geq 0, = 0 \iff x = y$
    \item $\rho(x, y) = \rho (y, x)$
    \item $\rho(x, y) \leq \rho (x, z) + \rho (y, z)$
  \end{enumerate}
\end{defn}
\begin{defn}$(X, \rho)$ --- \textbf{Metric space}.\end{defn}
\begin{defn}$x = \lim x_{n} \iff \rho(x_{n}, x) \to 0$\end{defn}
$X, \tau = \Set{G \subset X}$
\begin{defn}
  Let $X$ be arbitrary set. Then system of its subsets $\tau$ is called a
  \textbf{topology} if $\colon$
  \begin{enumerate}
  \item $\varnothing, X \in \tau$
  \item $G_\alpha \in \tau, \alpha \in \mathscr{A} \implies
    \bigcup\limits_\alpha G_\alpha \in \tau$
  \item $G_1, \dotsc, G_n \in \tau \implies \bigcap\limits_{j = 1}^n G_j \in
    \tau$
  \end{enumerate}
And any set $G \in \tau$ is called \textbf{open}.
\end{defn}

\begin{defn}$(X, \tau)$ --- \textbf{Topological space}.\end{defn}
$x = \lim x_n \quad \forall G \in \tau: x \in G \quad \exists N: \forall n > N \quad x_n \in G$\\
G --- open in $\tau$ \\
$F = X \setminus G$ --- closed
\begin{defn}
  $B_r (a) = \Set{x | \rho(x, a) < r}$ --- \textbf{open ball}
\end{defn}
\begin{stm}
  Any metric space gives rise to a topological space in a rather simple way.
  Let's call the subset $G \subset X$ open if and only if $\forall x \in G$
  there is some $r$ such that open ball $B_r(x)$ is contained in $G$. Then $\tau = \bigcup B_r (x)$
\end{stm}
\begin{stm}
  $b \in B_{r_1} (a_1) \cap B_{r_2} (a_2) \implies \exists r_3 > 0: B_{r_3}
  (a_3) \subset B_{r_1} (a_1) \cap B_{r_2} (a_2)$
\end{stm}
\noindent
\begin{ex}
  $\R, \rho(x, y) = |x - y|$, MS
\end{ex}
\begin{ex}
  $\bar{x} = (x_1, \dotsc, x_n) \in \R^n, \rho(\bar{x}, \bar{y}) = \sqrt{\sum\limits_{j = 1}^n(x_j - y_j)^2}, MS$
\end{ex}
\begin{ex}
    \hspace{-1.9em}$\begin{aligned}[t]
      \bar{x} &= (x_1, \dotsc, x_n, \dotso) \in \R^\infty \\
      \alpha\bar{x} &= (\alpha x_1, \dotsc, \alpha x_n, \dotsc) \\
      \bar{x} + \bar{y} &= (x_1 + y_1, \dotsc, x_n + y_n, \dotsc)
  \end{aligned}$

      Let's define $\alim{\bar{x}_m}{m \to \infty}$
      \begin{itemize}
          \item{in $\R^n$:}
              $\bar{x}_n \to \bar{x} \iff \forall j = 1, \dotsc, n\quad\ \ \:\,x_j^{(m)} \xrightarrow[m \to \infty]{} x_j$
          \item{in $\R^\infty$:}
              $\bar{x}_m \to \bar{x} \overset{def}{\iff} \forall j = 1,2,3,\dotso \quad x_j^{(m)} \xrightarrow[m \to \infty]{} x_j$
      \end{itemize}
\end{ex}
\begin{defn}
  $\rho(\bar{x}, \bar{y}) \defeq \sum\limits_{n = 1}^\infty \dfrac{1}{2^n}\underbrace{\dfrac{|x_n - y_n|}{1 + |x_n - y_n|}}_{\phi(|x_n - y_n|)}$
  --- \textbf{Urysohn metric}.

  \noindent
  \begin{minipage}{.65\linewidth}
    $\displaystyle\phi(t) = \dfrac{t}{1 + t} \\ \\
    \phi(t_1) + \phi(t_2) \geq \phi(t_1 + t_2) \\ \\
    \phi(t_1) + \phi(t_2) = \dfrac{t_1}{1 + t_1} + \dfrac{t_2}{1 + t_2} \geq \dfrac{t_1}{1 + t_1 + t_2} + \dfrac{t_2}{1 + t_1 + t_2} \\ \\
    \phi(t_1) + \phi(t_2) \geq \dfrac{t_1 + t_2}{1 + t_1 + t_2} = \phi(t_1 + t_2)$
  \end{minipage}%
  \begin{minipage}{.35\linewidth}
    \begin{tikzpicture}
      \begin{axis}[
        width = \linewidth,
        xmin  = 0,
        xmax  = 15,
        domain= 0:15,
        ymax  = 1.2,
        samples = 100,
        minor tick num = 0 ]
        \addplot[thick]{x/(1+x)};
      \end{axis}
    \end{tikzpicture}
  \end{minipage}
  \begin{stm}
      $\rho(\bar{x}_m, \bar{x}) \xrightarrow[m \to \infty]{} 0 \iff x_j^{(m)} \to x_j\ \forall j$
  \end{stm}
  \begin{proof}\leavevmode
    \begin{itemize}
      \item $\implies$

      $f(|x^{(n)}_k - x_k|) \leq 2^k \rho(x^{(n)}, x)$ \\
      Let $\rho(x^{(n)}, x) \le \dfrac{\epsilon}{2^k}$, then $f(|x^{(n)}_k - x_k|) < \epsilon$ \\
      $|x^{(n)}_k - x_k| = t = \dfrac{1}{1 - f(t)} - 1$, then $t \to 0$
      \item $\Leftarrow$

      Let's choose $k_0$ for which $\sum\limits_{k=k_0+1}^{\infty} \dfrac{1}{2^k} < \epsilon$ \\
      Let's choose $n_0$ for which $\forall k \leq k_0, n > n_0: |x_k^{(n)} - x_k| < \epsilon$. \\
      Then $\rho(x^{(n)}, x) < \sum\limits_{k=1}^{k_0} \dfrac{\epsilon}{2^k} + \epsilon < 2 \epsilon$ \\
      Letting  $\epsilon \to 0$, we get what we want \qedhere
    \end{itemize}
  \end{proof}
  In this way $\R^\infty$ is a metrizable space.
\end{defn}
\begin{ex}
$X, \rho(x, y) \defeq
\begin{cases}
    0, & x = y\\
    1, & x \neq y
\end{cases}$ --- \textbf{Discrete metric}. \\
$x_n \to x,\ \epsilon = \dfrac{1}{2},\ \exists M:\ m > M \implies \rho(x_m, x) < \dfrac{1}{2} \implies\\
\rho(x_m, x) = 0 \implies x_m = x$
\end{ex}
\begin{defn}
    \[(X, \tau);\ \forall A \subset X;\]
  \begin{align*}
     \inter{A}& \defeq \bigcup\limits_{G \subset A} G \textnormal{ is open}; \\
     \cl{A}& \defeq \bigcap\limits_{A \subset G} G \textnormal{ is closed}; \\
     \fr{A}& \defeq \cl{A} \setminus \inter{A}
  \end{align*}
\end{defn}
\noindent
$(X, \rho);$ Having a metric space one can describe closure of a set.
\begin{gather*}
    \rho(x, A) \defeq \inf\limits_{a \in A}\rho(x, a) \\
    \rho(A, B) \defeq \inf\limits_{\substack{a \in A\\b \in B}}\rho(a, b) \\
    \rho(x, A) = f(x), x \in X
\end{gather*}
\begin{stm}
  Function f(x) is continuous.
\end{stm}
\begin{proof}
  $\forall x, y \in X \\
  f(x) = \rho(x, A) \underset{\forall \alpha \in A}{\leq} \rho(x, \alpha) \leq \rho(x, y) + \rho(y, \alpha) \\
  \forall \epsilon > 0\ \exists \alpha_\epsilon \in A:\ \rho(y, \alpha_\epsilon) < \rho(y, A) + \epsilon = f(y) + \epsilon \\
  f(x) \leq f(y) + \epsilon + \rho(x, y),\ \epsilon \to 0 \\
  \begin{cases}
    f(x) \leq f(y) + \rho(x, y) \\
    f(y) \leq f(x) + \rho(x, y)
  \end{cases} \implies |f(x) - f(y)| \leq \rho(x, y)$
\end{proof}
\begin{stm}
  $x \in \cl{A} \iff \rho(x, A) = 0$
\end{stm}
Let's look at the metric spaces in terms of separation of sets from each other by open sets. \\
$x, y \\
r = \rho(x, y) > 0 \\
B_{\dfrac{r}{3}}(x),\ B_{\dfrac{r}{3}}(y)$ \\
In any metric space separability axiom is true.
\begin{thm}
  Any metric space is a normal space, \\i.e.
  $\forall\ closed\ disjoint\ F_1, F_2 \in X,\ \exists\ open\ disjoint\ G_1, G_2\colon F_j \in G_j,\ j = 1, 2$
\end{thm}
\begin{proof}
  $g(x) = \dfrac{\rho(x, F_1)}{\rho(x, F_1) + \rho(x, F_2)}$ --- continuous on X \\
  $x \in F_1,\ \cl{F_1} = F_1,\ \rho(x, F_1) = 0,\ g(x) = 0 \\
  x \in F_2,\ g(x) = 1$ \\
  Let's look at $(-\infty; \dfrac{1}{3}), (\dfrac{2}{3}, \infty)$ --- by continuity their inverse images under $g$ are open. \\
  $G_1 = g^{-1}(-\infty; \dfrac{1}{3}) \\
  G_2 = g^{-1}(\dfrac{2}{3}; \infty)$
\end{proof}
\begin{defn}
  Metric space is \textbf{complete} if $\rho(x_n, x_m) \to 0 \implies \exists x = \lim x_n \\
  \R^\infty$ -- complete (by completeness of the rational numbers). \\
  In complete metric spaces the nested balls principle is true.
\end{defn}
\begin{thm}
  X -- complete metric space, $\overline{V}_{r_n}$ -- system of closed balls.
  \begin{enumerate}
      \item $\overline{V}_{r_{n + 1}} \subset \overline{V}_{r_n}$ -- the system is nested.
      \item $r_n \to 0$
    \end{enumerate}
  \underline{Then:} $\bigcap\limits_n \overline{V}_{r_n} = \{a\}$
\end{thm}
\begin{proof}
  Let $b_n$ be centers of $\overline{V}_{r_n}, \\
  m \geq n,\ b_m \in \overline{V}_{r_n},\ \rho(b_m, b_n) \leq r_n \to 0\ \forall m \geq n \\
  \rho(b_m, b_n) \to 0 \xRightarrow{compl.} \exists a = \lim b_n$
  Since the balls are closed a $\in$ every ball. \\
  $r_n \to 0 \implies$ there is only one common point.
\end{proof}
\noindent
$(X, \tau)$ --- topological space
$\\A \subset X,\ \tau_a = \Set{G \cap A, G \in \tau}$ --- topology induced on $A$
\begin{defn}
  $X \textnormal{--- metric space},\ A \subset X,\ \cl{A} = X$ \\
  \underline{Then:} A -- \textbf{dense} in X \\
  \underline{If} $\inter{\cl{A}} = \varnothing$ A -- \textbf{nowhere dense} in X.
\end{defn}
\begin{note}
  It is easy to understand, that in metric spaces nowhere density means the following: \\
  $\forall \text{ ball } V\ \exists V^{'} \subset V\colon V^{'}$ contains no
  points from A.
\end{note}
\begin{defn}
  $X$ is called \textbf{first Baire category set}, if it can be written as at most
  countable union of $x_n$ each nowhere dense in $X$.
\end{defn}
\begin{thm}[Baire category theorem]
  Complete metric space is second Baire category set in itself.
\end{thm}
\begin{proof}
  Let $X$ be first Baire category set. \\
  $X = \bigcup\limits_n X_n \quad \forall \overline{V}\ X_1\ \textnormal{is nowhere dense}. \\
  \overline{V}_1 \subset \overline{V}\colon\ \overline{V}_1 \cap X_1\ = \varnothing \\
  X_2\ \textnormal{is nowhere dense}\ \overline{V}_2 \subset \overline{V}_1 \colon \overline{V}_2 \cap X_2 = \varnothing \\
  r_2 \leq \dfrac{r_1}{2} \\
  \vdots \\
  \{\overline{V}_n\},\ r_n \to 0,\ \bigcap\limits_n \overline{V}_n = \{a\},\ X =
  \bigcup X_n,\ \exists n_0 \colon a \in X_{n_0} \\
  X_{n_0} \cap \overline{V}_{n_0} = \varnothing \contr\ a \in \overline{V}_{n_0}$
\end{proof}
\begin{cor}
  Complete metric space without isolated points is uncountable.
\end{cor}
\begin{proof}
  No isolated points are present $\implies$ every point in the set is nowhere dense in it. Let $X$ be countable:
    $X = \bigcup\limits_n \{X_n\}$, then it is first Baire category set. $\contr$
\end{proof}
\begin{defn}
  $K$ --- \textbf{compact} if
  \begin{enumerate}
    \item $K = \cl{K}$
    \item \label{itm:second}$x_n \in K\ \exists n_1 < n_2 < \dotso\ x_{n_j} - \textnormal{converges in $X$}$.
  \end{enumerate}
  If only \ref{itm:second} is present, the set is called \textbf{precompact}.
\end{defn}
\begin{thm}[Hausdorff]
  Let $X$ --- metric space, $K$ --- closed in $X$. \\
  \underline{Then:} $K$ --- compact $\iff$ $K$ --- totally bounded, \\
  i.e. $\forall \epsilon > 0\ \exists a_1,\dotsc, a_p \in X \colon\ \forall b \in K\ \exists a_j \colon\ \rho(a_j, b) < \epsilon \\
  (a_1, \dotsc, a_p - finite\ \epsilon-net)$
\end{thm}
\begin{proof}\leavevmode
  \begin{itemize}
    \item Totally bounded $\implies$ compact

      $K$ is totally bounded, $x_n \in K\ n_1 < n_2 < \dotsb < n_k < \dotsb,$
      $x_n$ converges in $K$
      \[\epsilon_k \downarrow \to 0\ \epsilon_1 \quad K \subset \bigcup\limits_{j = 1}^p \overline{V}_j,\ rad = \epsilon_1\qquad \text{($\epsilon_1$-net)} \]
      $n$ is finite $\implies$ one ball will contain infinetely many $x_n$ elements.

      Let's look at $\overline{V}_{j_0} \cap K$ --- totally bounded $= K_1, \diam{K_1} \leq 2\epsilon_1$ \\
      $\epsilon_2\quad K_1 \subset \bigcup\limits_{j =1}^n
      \overline{V^{'}}_{j},\ rad = \epsilon_2.$

      Then one of $\overline{V'}$ contains infinitely many elements of the sequence contained in $K_1$.

      $\overline{V'}_{j_0} \cap K_1 = K_2,\ \diam{K_2} \leq 2\epsilon_2$ and so on.

      $K_n \supset K_{n+1} \supset K_{n+2} \supset \dotso,\ \diam{K_N} \leq 2\epsilon_n
      \xRightarrow{\text{by compl.}} \underbrace{\bigcap\limits_{n = 1}^\infty K_n}_{\diam{K_n} \to 0,\ \{x\}} \neq \varnothing$

      Take $x_{n_1}$ from $K_1$, $x_{n_2}$ from $K_2 \dotso$
    \item Compact $\implies$ totally bounded

      $K$ --- compact $\forall \epsilon\ \exists$ finite $\epsilon$-net? \\
      By contradiction: $\exists\epsilon_0 > 0\colon$\ finite $\epsilon_0$-net is impossible to construct. \\
    $\forall x_1 \in K\ \exists x_2 \in K \colon\ \rho(x_1, x_2) > \epsilon_0$ (or else system of $x_1$ --- finite $\epsilon$-net) \\
    $\{x_1, x_2\}$ - choose $x_3 \in K\colon \rho(x_3, x_i) > \epsilon_0,\ i = 1, 2$ and so on. \\
    $x_n \in K:\ n\ \neq m\ \rho(x_n, x_m) > \epsilon_0$ --- contains no converging subsequence $\implies$
    set is not a compact. $\contr$ \qedhere
  \end{itemize}
\end{proof}
\section{Normed spaces}
\begin{defn}
  $X$ --- \textbf{linear set}, $x + y, \alpha \cdot x$, $\alpha \in \R$ \\
  The purpose of norm definition, is to construct a topology on $X$, so that 2 linear operations are continuous on it.
\end{defn}
$\phi\colon X \to \R\colon$
\begin{enumerate}
\item $\phi(x) \geq 0,\ = 0 \iff x = 0$
\item $\phi(\alpha x) = |\alpha| \phi(x)$
\item $\phi(x + y) \leq \phi(x) + \phi(y)$
\end{enumerate}
\begin{defn}
  $\phi$ --- \textbf{norm} on $X$, $\phi(x) = \|x\|$
\end{defn}
$\rho(x, y) \defeq \|x - y\|$ --- metric on $X$.
\begin{defn}
  $\\(X, \|\cdot\|)$ --- \textbf{normed space} --- special case of metrical space.
\end{defn}
\noindent
$x = \lim x_n \overset{def}{\implies} \rho(x_n, x) \to 0 \iff \|x_n - x\| \to 0$
\begin{stm}
  In the topology of a normed space linear operations are continuous on $X$.
\end{stm}
\begin{proof}\leavevmode
  \begin{enumerate}
  \item
    $\begin{aligned}[t]
      x_n \to x,\ y_n \to y;\ \|(x_n + y_n) - (x + y)\| & = \|(x_n - x) + (y_n - y)\|  \leq \\
      & \leq  \tendsto{\|x_n - x\|}{0} + \tendsto{\|y_n - y\|}{0} \\
      & \implies x_n + y_n \to x + y
    \end{aligned}$
  \item
    $\begin{aligned}[t]
       \alpha_n \to \alpha,\ x_n \to x;\ \|\alpha_n x_n - \alpha x\| & =
        \|(\alpha_n - \alpha)x_n + \alpha(x_n - x)\| \leq \\
        & \leq \tendsto{|\alpha_n - \alpha|}{0} \cdot \underbrace{\|x_n\|}_{bounded} + \tendsto{\alpha\|x_n - x\|}{0}
    \end{aligned}$ \\
    $x_n \to x \implies \|x_n\|$ --- bounded. \\
    $\alpha_x x_n \to \alpha x$ \qedhere
  \end{enumerate}
\end{proof}
\begin{stm}
  From the triangle inequality $\bigl|\|x\| - \|y\|\bigr| \leq \|x - y\| \\
  x_n \to x \implies \|x_n\| \to \|x\|$ \\
  Norm is continious.
\end{stm}
\begin{ex}
  $\R^n$
  \begin{enumerate}
  \item $\|\bar{x}\| = \sqrt{\sum\limits_{k = 1}^n x_k^2}$
  \item $\|\bar{x}\|_1 \defeq \sum\limits_{k = 1}^n|x_k|$
  \item $\|\bar{x}\|_2 \defeq \max \{|x_1|, \dotsc, |x_n|\}$
  \item $C[a, b]$ --- functions continuous on $[a, b];\ \|f\| = \max\limits_{x \in [a, b]}|f(x)|$
  \item $L_p (E) = \Set{f \textnormal{ --- measurable},\ \int\limits_E \abs{f}^p < +
    \infty}\\
    p \geq 1,\ \|f\|_p = \biggl(\displaystyle\int\limits_E |f|^p\biggr)^{\dfrac 1p}$
  \end{enumerate}
\end{ex}
Because the set of points is the same, arises the question about convergence
comparison. \\
$\|\cdot\|_1 \sim \|\cdot\|_2,\ x_n \overset{\|\cdot\|_1}{\to} x \iff x_n \overset{\|\cdot\|_2}{\to} x$
\begin{stm}
  $\\\|\cdot\|_1 \sim \|\cdot\|_2 \iff \exists a, b > 0\colon \forall x \in X
  \implies a\|x_1\|_1 \leq \|x\|_2 \leq b \|x\|_1$
\end{stm}
\begin{thm}[Riesz]
  $X,\ \dim{X} < +\infty$ --- linear set. \\
  \underline{Then:} Any pair of norms in $X$ are equivalent.
\end{thm}
\begin{proof}
  $l_1, \dotsc, l_n$ --- linearly independent from $X$. $\forall x \in X =
  \sum\limits_{k = 1}^n \alpha_k l_K$
  \[\bar{x} \leftrightarrow (l_1, \dotsc, l_n) = \bar{l} \in \R^n\]
  Let $\|\cdot\|$ --- some norm in $X$.
  \begin{gather*}
  \|x\| \underset{\triangle}{\leq} \sum\limits_{k = 1}^n \|l_k\| |\alpha_k|
  \underset{\text{Cauchy}}{\leq} \underbrace{\sqrt{\sum\limits_{k=1}^n \|l_k\|^2}}_{const(B), B - basis}
  \equalto{\sqrt{\sum\limits_{k=1}^n |\alpha_k|^2}}{\|\bar{\alpha{}}\| = \|x\|_1} \\
  \|x\|_1 = \sqrt{\sum\limits_{k=1}^n \|\alpha_k\|^2},\ x = \sum \alpha_k l_k  \\
  \|x\| \leq b \|x\|_1 \\
  ?\exists a > 0\colon a\|x\|_1 \leq \|x\| \implies \|\cdot\| \sim \|\cdot\|_1 \\
  \end{gather*}
  Let $f(\alpha_1, \dotsc, \alpha_n) = \norm{\sum\limits_{k = 1}^n \alpha_k l_k}$
  \begin{align*}
    \\|f(\bar{\alpha} + \Delta\bar{\alpha}) - f(\bar\alpha)| = &
      \abs{\rule{0em}{2em}\norm{\sum_{k=1}^n
    \alpha_k l_k + \sum_{k=1}^n \Delta \alpha_k l_K } - \norm{\sum_{k=1}^n
    \alpha_k l_k}} \leq \\ & \norm{\sum_{k=1}^n \Delta \alpha_k l_k} \leq \tendsto{\sum
    \| l_k\| |\Delta \alpha_k|}{0, \Delta \alpha_k \to 0} \implies \text{$f$ is continuous on $\R^n$}
  \end{align*}

  $S_1 = \Set{\sum\limits_{k=1}^n \abs{\alpha_k}^2 = 1} \subset \R^m$, f --- continuous
  on $S_1$, $S_1$ --- compact, $\bar{\alpha}^* \in S_1$ \\
  By Weierstrass theorem there exists a point $\alpha^* \in S_1$ on a sphere,
  in which function $f$ achieves its minimum
  $\implies \forall \alpha \in S_1\ f(\bar{\alpha}^*) \leq f(\bar{\alpha})$

  If $f(\bar{\alpha}^*) = 0$,
  then $\norm{\sum\limits_{k=1}^n \alpha_k^* l_k} = 0 \implies
  \sum\limits_{k=1}^n \alpha_k^* l_k = 0,\ \bar{\alpha}^* \in S_1$, \\
  but $l_1 \dots l_n$ are linearly independent $\contr
  \implies \min\limits_{S_1} f = m > 0$
  \begin{align*}
    \|x\| = \norm{\sum_{k=1}^n \alpha_k l_k}  = f(\bar{\alpha}) =
    \sqrt{\sum_{k=1}^n \alpha_k^2} \cdot \norm{\sum
    \underset{\beta_k}{\boxed{\dfrac{\alpha_k}{\sqrt{\sum_{k=1}^n \alpha_k^2}}}}
    \cdot l_k} & \geq ,\ \bar{\beta} = (\beta_1 \dots \beta_n) \in S_1 \\
    & \geq m \cdot \|x\|_1,\ a = m \qedhere
  \end{align*}
\end{proof}
\begin{cor}
  $X$ --- NS, $Y \subset X, \dim{Y} < + \infty \implies Y = \cl{Y}$
  \begin{note}
    Functional analysis differentiates between linear subset (set of points,
    closed by addition and scalar multiplication) and linear subspace (closed
    linear subset).
  \end{note}
\end{cor}
\begin{proof}
  $Y = \L(l_1, \dotsc, l_n) = \Set{\sum\limits_{i=1}^n \alpha_il_i | l_1, \dotsc, l_n - \text{lin. indep.}} \\
  y_m \in Y, y_m \to y$ in $X \implies y \in Y?\\
  \norm{y_m - y} \to 0 \implies \norm{y_m - y_p} \to 0,\ m,p \to \infty \\
  \norm{y}, y \in Y.\\
  \textnormal{By Riesz theorem all norm in Y are equivalent.} \\
  y = \sum\limits_{j=1}^n \alpha_j l_J, \norm{y}_0 = \sqrt{\sum\limits_{j=1}^n
    \alpha_j^2}$ --- some norm (by linear independance). \\
  By Riesz theorem $\|y\| \sim \|y\|_0 \\
  \underbrace{\|y_m - y_p\|}_{\in Y} \to 0 \implies \|y_m - y_p\|_0 \to 0 \\
  \bar{\alpha}=(\alpha_1, \dotsc, \alpha_n) \in \R^n\ y_m = \sum\limits_{i = 1}^n \alpha_i^{(m)}l_i \\
  |\alpha_i^{(m)} - \alpha_i^{(l)}| \to 0\ \forall i = 1, \dotsc, n;\
  \bar{\alpha} = (\alpha_1^{(m)}, \dotsc, \alpha_n^{(m)}) \to \alpha^* =
  (\alpha_1^*, \dotsc , \alpha_n^*) \\
  y^* = \sum\limits_{i=1}^n \alpha_i^* l_I \in Y,\ \|y_m - y\| \to 0, y = y^*
  \implies y \in Y$
\end{proof}
\begin{defn}
  If normed space if complete, then it is called \textbf{B-space} or \textbf{Banach space}.
\end{defn}
\begin{ex}
 $C[a, b]$ --- functions continuous on $[a, b]$.
\end{ex}
\begin{ex}
  Lebesgue space,
  $p \geq 1, L_p(E) = \Set{f \textnormal{ is measurable on $E$},\ \int\limits_E \abs{f}^P < + \infty}$.
\end{ex}
If $X$ --- Banach space, \\
$\begin{aligned}[t]
& \sum_{n = 1}^\infty x_n = \lim_{n \to \infty}
\sum_{k = 1}^n x_k,\ \sum_{n=1}^\infty \|x_n\| < + \infty \\
& \|S_n - S_m\| = \norm{\sum_{k = m + 1}^n x_k} \leq \sum_{m + 1}^n
\|x_k\| \xrightarrow[n, m \to \infty]{} 0 \\
& \implies \|S_n - S_m\| \to 0 \implies  \exists \lim_{n \to \infty} S_n,
\sum_{k=1}^n x_k \text{--- converges.}
\end{aligned}$ \\
In Banach spaces works the theory of absolute convergence of numerical series.
\begin{lemma}[Riesz's lemma about almost perpendicular]
  $Y$ --- eigen subspace of $X$ --- normed space.
  $\forall \epsilon \in (0, 1)\ \exists z_\epsilon \in X \colon$
\begin{enumerate}
\item $\|z_\epsilon\| = 1$
\item $\rho(z_\epsilon, Y) > 1 - \epsilon$
\end{enumerate}
\end{lemma}
\begin{proof}
  $\exists x \notin Y\ d = \rho(x, Y),\ d = 0\ \exists y_n \in Y\colon \|x -
  y_n\| < \dfrac{1}{n},\ n \to \infty,\ y_n \to x \\
  Y = \cl{Y} \implies x \in Y \contr x \notin Y,\ d > 0 \\
  \forall \epsilon \in (0, 1)\ \dfrac{1}{1 - \epsilon} > 1\ \exists y_\epsilon
  \in Y\colon \|x - y_\epsilon\| < \dfrac{1}{1 - \epsilon}d \\
  z_\epsilon = \dfrac{x - y_\epsilon}{\|x - y_\epsilon\|},\ \|z_\epsilon\| = 1,\
  \forall y \in Y\ \|z_\epsilon - y\| = \norm{\dfrac{x - y_\epsilon}{\|x -
      y_\epsilon\|}} = \dfrac{\|x - \overbrace{(y_\epsilon + \|x - y_\epsilon\| \cdot y)}^{\in Y}\|
    \geq d}{\|x- y_\epsilon\| < \dfrac{1}{1 - \epsilon}d} > 1 - \epsilon$
\end{proof}
\begin{cor}
  $\dim{X} = + \infty$, $S$ --- sphere in $X$, $r_S = 1 \Set{x | \|x\| = 1}
  \implies S$ --- not a compact.
\end{cor}
\begin{proof}
  $\forall x_1 \in S,\ Y_1 = \L\{x_1\}$ --- finite dimensional linear set.
  $\implies\ \text{closed in}\ X \implies Y_1$ --- subspace. $\dim{X} = + \infty
  \implies Y_1$ --- eigen subspace. \\
  Then by the Riesz lemma:
  \begin{align*}
    & \exists x_2 \in S\colon \|x_2 - x_1\| > \dfrac 12 \\
    & Y_2 = \L\{x_1, x_2\}\ \exists x_3 \in S\colon \|x_3 - x_j\| > \dfrac 12,\ j = 1,2
  \end{align*}
  Continue by induction. Because $\dim{X} = + \infty$ the process willl never
  stop. \\
  $x_n \in S \colon \|x_n - x_m\| > \dfrac 12,\ n \neq m$ --- obviously we
  cannot extract converging subsequence. $\implies S$ --- not a compact.
\end{proof}
\section{Inner product (unitary) spaces}
\begin{defn}
  $X$ --- linear space. \\
  $\phi \colon X \times X \to \R$
  \begin{enumerate}
      \item $\phi(x, x) \ge 0,\quad \phi(x, x) = 0 \iff x = 0$
  \item $\phi(x, y) = \phi(y, x)$
  \item $\phi(\alpha x + \beta y, z) = \alpha \phi(x, z) + \beta \phi(y, z)$
  \end{enumerate}
  $\phi$ --- \textbf{inner product}. \\
  $\phi(x, y) = \langle x, y \rangle$
\end{defn}
\begin{defn}
  $(X, \langle \cdot, \cdot \rangle)$ --- \textbf{inner product space}.
\end{defn}
\begin{ex}
  $\R^n, \langle \bar{x}, \bar{y} \rangle = \sum\limits_{j = 1}^n x_j y_j$
\end{ex}
\begin{stm}[Schwarz]
  $\forall x, y \in X \quad \abs{\langle x, y \rangle} \leq \sqrt{\langle x,
    x \rangle} \cdot \sqrt{\langle y, y \rangle}$
\end{stm}
\begin{proof}
  $\begin{aligned}[t]
    & \lambda \in \R \\
    & f(\lambda) =
      \equalto{\langle \lambda x + y, \lambda x + y \rangle}
      {\lambda^2 \langle x, x \rangle + 2 \lambda \langle x, y \rangle + \langle y, y \rangle} \geq 0 \\
    & D = 4 \langle x, y \rangle^2 - 4 \langle x, x \rangle \cdot \langle y, y
    \rangle \leq 0 \qedhere
  \end{aligned}$
\end{proof}


\begin{cor}[Cauchy inequality for sums]
  Consider $X = \R^n, \norm{x} \defeq \sqrt{\langle x, x \rangle}$. Then
  \[
  \norm{x+y}^2 = \inprod{x+y, x+y} = \norm{x}^2 + 2 \cdot \!\!\underbracket{\inprod{x, y}}_{{} \le \norm{x} \cdot \norm{y}}\!\! + \norm{y}^2 \le \bigl(\norm{x} + \norm{y}\bigr)^2
  \]
\end{cor}

Any inner product space is a special case of a normed space. The specifics is that we can measure the angles between points:
\[
  x \perp y \iff \inprod{x, y} = 0
\]
In this case the Pythagorean theorem takes place:
\[
\norm{x+y}^2 = {\norm x}^2 + {\norm y}^2
\]

\noindent In inner product spaces the parallelogram law plays a significant role:
\[
\norm{x+y}^2 + \norm{x-y}^2 = 2 \norm{x}^2 + 2 \norm{y}^2 \quad \forall x, y \in X
\]

% не хуйня ли часом с этого места?
%Да вроде нет, In an inner product space, the norm is determined using the inner
%product: \|x\|^2=\langle x, x\rangle.\ (c) https://en.wikipedia.org/wiki/Parallelogram_law
In an inner product space norm is determined by inner product:
$\|x\|^2 = \inprod{x, x}$ \\
It can be proved that if parallelogram law holds, then the norm must be
determined by some inner product. Let $X$ be some normed space, $x \in X$, then
$\inprod{\cdot, \cdot} \mapsto \norm{x} = \sqrt{\inprod{x, x}}$.
For any norm satisfying the parallelogram law, the inner
product generating the norm is unique.

\begin{ex}
  $C_{[a, b]},\ \|f\| = \max\limits_{x \in [a, b]}|f(x)|$, $\|f\|$ doesn't satisfy
  the parallelogram law and thus is not determined by any inner product. This
  fact implies that $C_{[a, b]}$ is not an inner product space.
\end{ex}

\begin{defn}
  \textbf{Orthonormal set} --- a set of points $\Set{l_1, l_2, \dotsc}$ (may be finite):
  \begin{enumerate}
    \item $\norm{l_i} = 1$
    \item $l_i \perp l_j, \quad i \ne j$
  \end{enumerate}
\end{defn}
\noindent Every orthonormal set is linearly independent.

\begin{defn}
  Let $x \in X, \Set{l_i}$ --- ONS. Then \\
  $\inprod{x, l_j}$ --- \textbf{Fourier coefficient}, \\
  $\sum\limits_j \inprod{x, l_j} l_j$ --- \textbf{Fourier series} of point $x$.
\end{defn}
\noindent Fourier series is a special case of orthogonal series.

\begin{defn}
  $\sum\limits_j x_j$ --- orthogonal series $\iff x_i \perp x_j, \quad i \ne j$
\end{defn}

\noindent Let $\sum\limits_{j=1}^\infty x_j, S_m = \sum\limits_{j=1}^m x_j$. Then
\[\textstyle
\norm{S_m}^2 = \bigl\langle \sum\limits_{j=1}^m x_j,\sum\limits_{j=1}^m x_j \bigr\rangle = \sum\limits_{j=1}^m \norm{x_j}^2
\]

\noindent This fact allows us to effectively build the theory of orthogonal series.

An important problem is concerned with Fourier series. Let $X$ is a normed space, $Y$ is a subspace of $X$,
\[
\forall x \in X \quad E_Y(x) = \rho(x, Y) = \inf_{y \in Y} \norm{x-y}
\]

\begin{defn}
  $E_Y (x)$ --- \textbf{best approximation} of point $x$ with points of the subspace $Y$, if $\exists y^* \in Y \quad E_y(x) = \norm{x - y^*}$ --- then $y^*$ is an element of best approximation.
\end{defn}

\begin{thm}[Borel]
  $\dim Y < +\infty \implies \forall x \in X \quad \exists y^* \in Y$ --- element of best approximation.
\end{thm}
\begin{proof}
  $Y = \L(\underbracket{l_1, l_2, \dotsc, l_n}_{\text{lin. indep.}})$

  Consider $f(\alpha_1, \dotsc, \alpha_n) = \bigl\| x - \sum\limits_{k=1}^n \alpha_k l_k \bigl\| \to \min$. By the triangle inequality for norm, $f(\bar \alpha)$ is continous on $\R^n, f \ge 0, E_Y(x) = \inf f(\bar \alpha)$.
  It is easy to find that there always is a ball $B(0, r) \subset \R^n$, outside
  of which $f > 2E_Y(x)$. So, $E_Y(x)$ is somewhere inside. But $f$ is
  continuous, the ball is compact, so, by the Weierstrass theorem, the minimum
  exists and it is located on the sphere $S(0, r)$.
\end{proof}

For abstract Fourier series the Borel theorem can be significantly strengthened by specifying the best approximation element.

\begin{thm}[extreme quality of Fourier series' partial sums]
    $\Set{e_j}$ --- ONS in $X$ \\
    $H_n = \L(l1, \dotsc, l_n)$ \\
    $E_{H_n}(x), S_n(x) = \sum\limits_{j=1}^n \inprod{x, l_j} l_j \implies E_{H_n}(x) = \norm{x-S_n(x)}$
\end{thm}
\begin{proof}
  $y = \sum\limits_{j=1}^n \alpha_j l_j \in H_n$
  \begin{align*}
    \norm{x-y}^2 &= \inprod{x - \sum \alpha_j l_j, x - \sum \alpha_j l_j} = \norm{x}^2 - 2 \sum \alpha_j \inprod{x, l_j} + \sum{\alpha_j^2} = \\
    {} &= {\underbracket{\norm{x}}_{\mathrm{const}}}^2 + \sum (\alpha_j - \inprod{x, l_j})^2 - \underbracket{\sum \inprod{x, l_j}^2}_{\mathrm{const}} \to \min
  \end{align*}
  \noindent So, the sum goes to minimum when the second summand is minimal. Obviously, it's minimal when $\forall (\alpha_j - \inprod{x, l_j}) = 0$. $E_Y(x)$ --- Fourier sum.
\end{proof}

\begin{cor}[Bessel's inequality]
  $\sum\limits_j \inprod{x, l_j}^2 \le \norm{x}^2$
\end{cor}
\begin{proof}
  $ 0 \le \norm{x - y^*}^2 = \norm{x}^2 - \sum\limits_j \inprod{x, l_j}^2 $
\end{proof}
\begin{cor}
  The series of Fourier coefficients' squares always converges
\end{cor}
% \section{Hilbert space.}
% \begin{defn}
%   \textbf{Hilbert space} --- complete, infinite dimensional, inner product space.
% \end{defn}
% \begin{ex}
%   $L_2(E)$ --- Hilbert space. \\
%   $\inprod{f, g} = \int\limits_E f \cdot g\ d\mu$ \\
%   $l_2 = \Set{(x_1, \dotsc, x_n, \dotsc) | \sum\limits_{n = 1}^\infty x_n^2 < +\infty}$ \\
%   $\inprod{\bar{x}, \bar{y}} = \sum\limits_{n = 1}^\infty x_n y_n,\ E = \N,\ \mu\{m\} = 1$
% \end{ex}
% \noindent
% When we have completeness we can define orthonormal basis.
% \begin{defn}
%   $H,\ \{l_n\}$ --- ONS $\colon \forall x = \sum\limits_{n = 1}^\infty \alpha_n l_n
%   \implies \inprod{x, l_m} = \sum\limits_{n = 1}^\infty \alpha_n \inprod{l_n, l_m} = \alpha_m$ \\ 
% In this sense basis decomposition is always a Fourier series.
%   \begin{enumerate}
%   \item Complete ONS\@: $H = \cl{\L\{l_1, l_2, \dots\}}$
%   \item Closed ONS\@: $\forall m\ \inprod{x, l_m} = 0 \implies x = 0$.
%   \end{enumerate}
% \end{defn}
% \begin{stm}
%   In Hilbert spaces two of the properties outlined above are equivalent. \\
%   $\sum\limits_{n = 1}^\infty y_n$ in $H$ --- orthogonal series. \\
%   $\|S_n - S_m\|^2 = \sum\limits_n^m \|y_n\|^2 \\
%   \|S_n - S_m\| \to 0 \iff \sum\limits_n^m \|y_k\|^2 \to 0$ \\
%   in H $\sum\limits_1^\infty \|y_k\|^2 < + \infty$
% \end{stm}
% \noindent
% ONS $x \mapsto \sum\limits_{k = 1}^\infty \inprod{x, l_k} l_k$ \\
% $\sum\limits_{k = 1}^\infty \inprod{x, l_k}^2 \leq \|x\|^2 < +\infty$
% \begin{stm}
%   In Hilbert space Fourier series converges for any point.
% \end{stm}                                    
% \begin{proof}
%   Complete ONS. \\
%   $\forall x \in H \forall \epsilon > 0\ \exists \sum\limits_{j =
%     1}^p \alpha_{kj}l_{kj}\colon \underbrace{\|x - \sum\limits_{j = 1}^p
%     \alpha_{kj}l_{kj}\|^2}_{\geq \|x - \sum\limits_{j = 1}^{k_p} \inprod{x, l_j} l_j\|^2}
%   \leq \epsilon^2$ \\
%   $S_m(x) by extremety \|x - S_{m + p}(x)\|^2 \leq \|x - S_m(x)\|^2$ \\
%   Implies partial sums go to x, $x = \sum\limits_{j = 1}^\infty \inprod{x, l_j} l_j$
%   If all fourier coefficients are zero, it means the ONS is closed $(x = 0)$. \\
%   Closed ONS. \\
%   $y = \sum\limits_{j = 1}^\infty \underbrace{\inprod{x, l_j}}_{ = \inprod{y,
%       l_j}} l_j \implies \inprod{y, l_j} = \inprod{x, l_j} \implies \inprod{y - x,
%     l_j} = 0$ \\
%   Because ONS is closed $y - x = 0,\ y = x$. Thus we can decompose any point into
%   Fourier series, and this implies that ONS is complete.
% \end{proof}
% Considering basis existance.
% \begin{defn}
%   Topological space is called \textbf{separable} if there exists countable dense set in it. \\
%   $X = \cl{\{a_1, \dotsc, a_n, \dotsc\}}$
% \end{defn}
% $H$ --- separable, $a_1, \dotsc, a_n, \dotsc$ We can orthogonalize these dots
% (Gramm-Shmidt), and we will get complete ONS.
% This means space separability is equivalent to basis existance.
% \begin{thm}[about best approximation in H]
%   $H$ --- HS, $M$ --- closed convex subset of H, then $\forall x \in H \exists!
%   y \in M \colon \|x -y\| = \inf\limits_{z \in M}\|x - z\|$.
%   M has elemet of best approzimation for any x from X, and only one.
% \end{thm}
% \begin{proof}
%   $d = \inf\limits_{z \in M}\|x - z\|$ by definition of infimum \\
%   $\forall n \in \N \exists y_n \in M \colon d \leq \|x - y_n\| < d + dfrac 1n$ \\
%   $\exitst ? y = \lim y_n \in M\ d \leq \|x - y\| \leq d$ \\
%   $y_n, y_m \in M$ --- convex. Implies $\dfrac{y_n + y_m}{2} \in M \implies d^2
%   \leq \norm{\dfrac{y_n + y_m}{2} - x}^2 = \dfrac{1}{4}\norm{\underbrace{(y_n - x)}_{z_1} + \underbrace{(y_m -
%     x)}_{z_2}}^2$ 
% \\ Let's use parallelogram law. $\|z_1 + z_2\|^2 + \|z_1 - z_2\|^2 = 2\|z_1\|^2
% + 2\|z_2\|^2$, $\underbrace{\|(y_n - x) + (y_m - x)\|^2}_{\geq 4d^2} + \|y_n -
% y_m\|^2 = 2\overbrace{\|y_n - x\|^2}^{\leq (d + \dfrac 1n)^2}
% + 2 \overbrace{\|y_m - x\|^2}^{\leq (d + \dfrac 1m)^2}$ \\
% $\|y_n - y_m\|^2 \leq 2(d + \dfrac 1n)^2 + 2(d + \dfrac 1m)^2 - 4d^2 = 4d \dfrac
% 1n + \dfrac{2}{n^2} + 4 d \dfrac 1m + \dfrac{2}{m^2} \xrightarrow{n, m \to
%   0}{0}$ \\
% $\|y_n - y_m\| \xrightarrow{n, m \to 0}{0} \implies \exists \lim y_n$
% \end{proof}
% \begin{cor}
%   $H$ --- HS, $H_1$ --- subspace (closed linear subset). $H_2$ = $H_1^{\perp} =
%   \Set{y \in H | y \perp x, x \in H_1}$ --- \textbf{orthogonal addition}. \\
%   $\forall x \in H$ can be однозначно. written as $x = x_1 + x _2,\ x_1 \in H_1,\ x_2 \in H_1^{\perp}$
% \end{cor}
% \begin{note}
%   $H = H_1 \oplus H_1^{\perp}$
% \end{note}


