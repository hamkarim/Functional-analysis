\section{Vector spaces.}
\subsection{Metric spaces.}

X, $\rho\colon X \times X \to \R_+$
\begin{defn}$\rho$ --- \textbf{metric}
  \begin{enumerate} 
    \item $\rho(x, y) \geq 0, = 0 \iff x = y$
    \item $\rho(x, y) = \rho (y, x)$
    \item $\rho(x, y) \leq \rho (x, z) + \rho (y, z)$
  \end{enumerate}
\end{defn}
\begin{defn}$(X, \rho)$ --- \textbf{Metric space}.\end{defn}
\begin{defn}$x = \lim x_{n} \iff \rho(x_{n}, x) \rightarrow 0$\end{defn}
\vspace{5mm}
$X, \tau = \{G \subset X\}$
\begin{enumerate}
  \item $\varnothing, X \in \tau$
  \item $G_\alpha \in \tau, \alpha \in \mathscr{A} \implies \bigcup\limits_\alpha G_\alpha \in \tau$
  \item $G_1, \dotsc, G_n \in \tau \implies \bigcap\limits_{j = 1}^n G_j \in \tau$
\end{enumerate}
\begin{defn}$(X, \tau)$ --- \textbf{Topological space}.\end{defn}
$x = \lim x_n \quad \forall G \in \tau: x \in G \quad \exists N: \forall n > N \implies x_n \in G$\\
G --- open in $\tau$ \\
$F = X \setminus G$ --- closed
\begin{defn}
  $B_r (a) = \{x\mid \rho(x, a) < r\}$ --- \textbf{open ball}
\end{defn}
\noindent
$\tau = \bigcup B_r (a)$
\begin{stm}
  $b \in B_{r_1} (a_1) \cap B_{r_2} (a_2) \implies \exists r_3 > 0: B_{r_3}
  (a_3) \subset B_{r_1} (a_1) \cap B_{r_2} (a_2)$
\end{stm}
\noindent
In this sense metric space is just a special case of topological space.
\begin{ex}
  $\R, \rho(x, y) = |x - y|$, MS
\end{ex}
\begin{ex}
  $\bar{x} = (x_1, \dotsc, x_n) \in \R^n, \rho(\bar{x}, \bar{y}) = \sqrt{\sum\limits_{j = 1}^n(x_j - y_j)^2}, MS$
\end{ex}
\begin{ex}
  \begin{align*}
    & \bar{x} = (x_1, \dotsc, x_n, \dotso) \in \R^\infty \\
    & \alpha\bar{x} = (\alpha x_1, \dotsc, \alpha x_n, \dotso) \\
    & \bar{x} + \bar{y} = (x_1 + y_1, \dotsc, x_n + y_n, \dotso)\\
    & \alim{\bar{x}_m}{m \to \infty}?\\
    & \textnormal{in}\ \R^n\hphantom{{}^\infty}\bar{x}_n \to \bar{x} \iff \forall j = 1,
      \dotsc, n\quad\ \ \:\,x_j^{(m)} \xrightarrow[m \to \infty]{} x_j \\
    & \textnormal{in}\ \R^{\infty}\hphantom{{}^n}\bar{x}_m \to \bar{x} \overset{def}{\iff} \forall j = 1,2,3,\dotso \quad x_j^{(m)} \xrightarrow[m \to \infty]{} x_j
  \end{align*}
\end{ex}
\begin{defn}
  $\rho(\bar{x}, \bar{y}) \defeq \sum\limits_{n = 1}^\infty \frac{1}{2^n}\underbrace{\frac{|x_n - y_n|}{1 + |x_n - y_N|}}_{\phi(|x_n - y_n|)}$
  --- \textbf{Urysohn metric}. \\
  $\phi(t) = \frac{t}{1 + t} \\
  \phi(t_1 + t_2) \leq \phi(t_1) + \phi(t_2) \\
  \rho(\bar{x}_m, \bar{x}) \xrightarrow{m \to \infty} 0 \iff x_j^{(m)} \to x_j\ \forall j$\\
  In this way $\R^\infty$ is a metrizable space.
\end{defn}
\begin{ex}
$X, \rho(x, y) \defeq
\begin{cases}
  0, x = y
  \\1, x \neq y
\end{cases}$ --- \textbf{Discrete metric}. \\
$x_n \to x,\ \epsilon = \frac{1}{2},\ \exists M:\ m > M \implies \rho(x_m, x) < \frac{1}{2} \implies\\
\rho(x_m, x) = 0 \implies x_m = x$
\end{ex}
\begin{defn}
  $\begin{aligned}[t]
    & (X, \tau);\ \forall A \subset X; \\
    & \inter{A} \defeq \bigcup\limits_{G \subset A} G - \textnormal{open}; \\
    & \cl{A} \defeq \bigcap\limits_{A \subset G} G - \textnormal{closed}; \\
    & \fr{A} = \cl{A} \setminus \inter{A}
  \end{aligned}$
\end{defn}
\noindent
$(X, \rho);$ Having a metric space one can describe closure of a set. \\ 
$\rho(x, A) \defeq \inf\limits_{a \in A}\rho(x, a) \\
\rho(A, B) = \inf\limits_{\substack{a \in A\\b \in B}}\rho(a, b) \\
\rho(x, A) = f(x), x \in X$
\begin{stm}
  Function f(x) is continuous.
\end{stm}
\begin{proof}
  $\forall x, y \in X \\
  f(x) = \rho(x, A) \underset{\forall \alpha \in A}{\leq} \rho(x, \alpha) \leq \rho(x, y) + \rho(y, \alpha) \\
  \forall \epsilon > 0\ \exists \alpha_\epsilon \in A:\ \rho(y, \alpha_\epsilon) < \rho(y, A) + \epsilon = f(y) + \epsilon \\
  f(x) \leq f(y) + \epsilon + \rho(x, y),\ \epsilon \to 0 \\
  \begin{cases}
    f(x) \leq f(y) + \rho(x, y) \\
    f(y) \leq f(x) + \rho(x, y)
  \end{cases} \implies |f(x) - f(y)| \leq \rho(x, y)\qedhere$
\end{proof}
\begin{stm}
  $x \in \cl{A} \iff \rho(x, A) = 0$
\end{stm}
Let's look at the metric spaces in terms of separation of sets from each other by open sets. \\
$x, y \\
r = \rho(x, y) > 0 \\
B_{\frac{r}{3}}(x),\ B_{\frac{r}{3}}(y)$ \\
In any metric space separability axiom is true.
\begin{thm}
  Any metric space is a normal space, \\i.e.
  $\forall\ closed\ disjoint\ F_1, F_2 \in X,\ \exists\ open\ disjoint\ G_1, G_2\colon F_j \in G_j,\ j = 1, 2$
\end{thm}
\begin{proof}
  $g(x) = \frac{\rho(x, F_1)}{\rho(x, F_1) + \rho(x, F_2)}$ --- continuous on X \\
  $x \in F_1,\ \cl{F_1} = F_1,\ \rho(x, F_1) = 0,\ g(x) = 0 \\ 
  x \in F_2,\ g(x) = 1$ \\
  Let's look at $(-\infty; \frac{1}{3}), (\frac{2}{3}, \infty)$ --- by continuity their inverse images under g are open. \\
  $G_1 = g^{-1}(-\infty; \frac{1}{3}) \\
  G_2 = g^{-1}(\frac{2}{3}; \infty)\qedhere$
\end{proof}
\begin{defn}
  Metric space is \textbf{complete} if $\rho(x_n, x_m) \to 0 \implies \exists x = \lim x_n \\
  \R^\infty$ -- complete (by completeness of the rational numbers). \\
  In complete metric spaces the nested balls principle is true.
\end{defn}
\begin{thm}
  X -- complete metric space, $\overline{V}_{r_n}$ -- system of closed balls. \\
  \begin{enumerate}
      \item $\overline{V}_{r_{n + 1}} \subset \overline{V}_{r_n}$ -- the system is nested.
      \item $r_n \to 0$
    \end{enumerate}
  \underline{Then:} $\bigcap\limits_n \overline{V}_{r_n} = \{a\}$
\end{thm}
\begin{proof}
  Let $b_n$ be centers of $\overline{V}_{r_n}, \\
  m \geq n,\ b_m \in \overline{V}_{r_n},\ \rho(b_m, b_n) \leq r_n \to 0\ \forall m \geq n \\
  \rho(b_m, b_n) \to 0 \xRightarrow{compl.} \exists a = \lim b_n$
  Since the balls are closed a $\in$ every ball. \\
  $r_n \to 0 \implies$ there is only one common point $\qedhere$.
\end{proof}
\noindent
$(X, \tau)$ --- topological space
$\\A \subset X,\ \tau_a = \{G \cap A, G \in \tau\}$ --- topology induced on $A$
\begin{defn}
  $X \textnormal{--- metric space},\ A \subset X,\ \cl{A} = X$ \\
  \underline{Then:} A -- \textbf{dense} in X \\
  \underline{If} $\inter{\cl{A}} = \varnothing$ A -- \textbf{nowhere dense} in X. \\
  It is easy to understand, that in metric spaces nowhere density means the following: \\
  $\forall \text{ ball } V\ \exists V^{'} \subset V\colon V^{'}$ contains no points from A.
\end{defn}
\begin{defn}
  $X$ is called \textbf{first Baire category set}, if it can be written as at most
  countable union of $x_n$ each nowhere dense in $X$.
\end{defn}
\begin{thm}[Baire category theorem]
  Complete metric space is second Baire category set in itself.
\end{thm}
\begin{proof}
  Let $X$ be first Baire category set. \\
  $X = \bigcup\limits_n X_n \quad \forall \overline{V}\ X_1\ \textnormal{is nowhere dense}. \\
  \overline{V}_1 \subset \overline{V}\colon\ \overline{V}_1 \cap X_1\ = \varnothing \\
  X_2\ \textnormal{is nowhere dense}\ \overline{V}_2 \subset \overline{V}_1 \colon \overline{V}_2 \cap X_2 = \varnothing \\
  r_2 \leq \frac{r_1}{2} \\
  \vdots \\
  \{\overline{V}_n\},\ r_n \to 0,\ \bigcap\limits_n \overline{V}_n = \{a\},\ X =
  \bigcup X_n,\ \exists n_0 \colon a \in X_{n_0} \\
  X_{n_0} \cap \overline{V}_{n_0} = \varnothing \to\leftarrow\ a \in \overline{V}_{n_0} \qedhere$
\end{proof}
\begin{cor}
  Complete metric space without isolated points is uncountable.
\end{cor}
\begin{proof}
  No isolated points are present $\implies$ every point in the set is nowhere dense in it. Let $X$ be countable:
    $X = \bigcup\limits_n \{X_n\}$, then it is first Baire category set. $\to\leftarrow\qedhere$
\end{proof}
\begin{defn}
  $K$ --- \textbf{compact} if
  \begin{enumerate}
    \item $K = \cl{K}$
    \item \label{itm:second}$x_n \in K\ \exists n_1 < n_2 < \dotso\ x_{n_j} - \textnormal{converges in $X$}$.
  \end{enumerate}
  If only \ref{itm:second} is present, the set is called \textbf{precompact}.
\end{defn}
\begin{thm}[Hausdorff]
  Let $X$ --- metric space, $K$ --- closed in $X$. \\
  \underline{Then:} $K$ --- compact $\iff$ $K$ --- totally bounded, \\
  i.e. $\forall \epsilon > 0\ \exists a_1,\dotsc, a_p \in X \colon\ \forall b \in K\ \exists a_j \colon\ \rho(a_j, b) < \epsilon \\
  (a_1, \dotsc, a_p - finite\ \epsilon-net)$
\end{thm}
\begin{proof}
  $\\\implies \\
  K \textnormal{--- totally bounded},\ x_n \in K\ n_1 < n_2 < \dotso < n_k < \dotso \\
  x_n - \textnormal{converges in K} \\
  \epsilon_k \downarrow \to 0\ \epsilon_1 \quad K \subset \bigcup\limits_{j = 1}^p \overline{V}_j,\ rad = \epsilon_1\ (\epsilon_1 - net) \\
  \textnormal{n is finite }\implies \textnormal{one ball will contain infinetely many}\ x_n\ \textnormal{elements}. \\
  \textnormal{Let's look at}\ \overline{V}_{j_0} \cap K\ \textnormal{--- totally bounded}\ = K_1,\ \diam{K_1} \leq 2\epsilon_1 \\
  \epsilon_2\quad K_1 \subset \bigcup\limits_{j =1}^n
  \overline{V^{'}}_{j},\ rad = \epsilon_2.\ \\
  \textnormal{Then one of}\ \overline{V^{'}}\ \textnormal{contains infinitely many elements of the sequence contained in}\ K _1 \\
  \overline{V^{'}}_{j_0} \cap K_1 = K_2,\ \diam{K_2} \leq 2\epsilon_2\ \textnormal{and so on}. \\
  K_n \supset K_{n+1} \supset K_{n+2} \supset \dotso,\ \diam{K_N} \leq 2\epsilon_n
  \xRightarrow{by\ space\ compl.} \underbrace{\bigcap\limits_{n = 1}^\infty K_n}_{\diam{K_n} \to 0,\ \{x\}} \neq \varnothing$ \\
  Take $x_{n_1}$ from $K_1$, $x_{n_2}$ from $K_2 \dotso$ \\
  $\impliedby$ \\
  $K$ --- compact $\forall \epsilon\ \exists$ finite $\epsilon$ --- net? \\
  By contradiction: $\exists\epsilon_0 > 0\colon$\ finite $\epsilon_0$-net is impossible to construct. \\
  $\forall x_1 \in K\ \exists x_2 \in K \colon\ \rho(x_1, x_2) > \epsilon_0$ (or else system of $x_1$ --- finite $\epsilon$-net) \\
  $\{x_1, x_2\}$ - choose $x_3 \in K\colon \rho(x_3, x_i) > \epsilon_0,\ i = 1, 2$ and so on. \\
  $x_n \in K:\ n\ \neq m\ \rho(x_n, x_m) > \epsilon_0$ --- contains no converging subsequence $\implies$ 
  set is not a compact. $\to\leftarrow\qedhere$
\end{proof}
\subsection{Normed spaces}
\begin{defn}
  $X$ --- \textbf{linear set}, $x + y, \alpha \cdot x$, $\alpha \in \R$ \\
  The purpose of norm definition, is to construct a topology on $X$, so that 2 linear operations are continuous on it.
\end{defn}
$\phi\colon X \to \R\colon$
\begin{enumerate}
\item $\phi(x) \geq 0,\ = 0 \iff x = 0$
\item $\phi(\alpha x) = |\alpha| \phi(x)$
\item $\phi(x + y) \leq \phi(x) + \phi(y)$
\end{enumerate}
\begin{defn}
  $\phi$ --- \textbf{norm} on $X$, $\phi(x) = \|x\|$
\end{defn}
$\rho(x, y) \defeq \|x - y\|$ --- metric on $X$. 
\begin{defn}
  $\\(X, \|\cdot\|)$ --- \textbf{normed space} --- special case of metrical space.
\end{defn}
\noindent
$x = \lim x_n \overset{def}{\implies} \rho(x_n, x) \to 0 \iff \|x_n - x\| \to 0$
\begin{stm}
  In the topology of a normed space linear operations are continuous on $X$.
\end{stm}
\begin{proof}
  \begin{enumerate}
  \item 
    $\begin{aligned}[t]
      x_n \to x,\ y_n \to y;\ \|(x_n + y_n) - (x + y)\| & = \|(x_n - x) + (y_n - y)\|  \leq \\ 
      & \leq  \tendsto{\|x_n - x\|}{0} + \tendsto{\|y_n - y\|}{0} \\
      & \implies x_n + y_n \to x + y
    \end{aligned}$
  \item 
    $\begin{aligned}[t]
       \alpha_n \to \alpha,\ x_n \to x;\ \|\alpha_n x_n - \alpha x\| & =
        \|(\alpha_n - \alpha)x_n + \alpha(x_n - x)\| \leq \\
        & \leq \tendsto{|\alpha_n - \alpha|}{0} \cdot \underbrace{\|x_n\|}_{bounded} + \tendsto{\alpha\|x_n - x\|}{0}
    \end{aligned}$ \\
    $x_n \to x \implies \|x_n\|$ --- bounded. \\
    $\alpha_x x_n \to \alpha x\qedhere$
  \end{enumerate}
\end{proof}
\begin{stm}
  From the triangle inequality $|\|x\| - \|y\|| \leq \|x - y\| \\
  x_n \to x \implies \|x_n\| \to \|x\|$ \\
  Norm is continious.
\end{stm}
\begin{ex}
  $\R^n$ 
  \begin{enumerate}
  \item $\|\bar{x}\| = \sqrt{\sum\limits_{k = 1}^n x_k^2}$
  \item $\|\bar{x}\|_1 \defeq \sum\limits_{k = 1}^n|x_k|$
  \item $\|\bar{x}\|_2 \defeq \max \{|x_1|, \dotsc, |x_n|\}$
  \item $C[a, b]$ --- functions continuous on $[a, b];\ \|f\| = \max\limits_{x \in [a, b]}|f(x)|$ 
  \item $L_p (E) = \{f - \textnormal{measurable},\ \int\limits_E |f|^p < +
    \infty\}\\
    p \geq 1,\ \|f\|_p = \Bigl(\displaystyle\int\limits_E |f|^p\Bigr)^{\frac 1p}$
  \end{enumerate}
\end{ex}
Because the set of points is the same, arises the question about convergence
comparison. \\
$\|\cdot\|_1 \sim \|\cdot\|_2,\ x_n \overset{\|\cdot\|_1}{\to} x \iff x_n \overset{\|\cdot\|_2}{\to} x$
\begin{stm}
  $\\\|\cdot\|_1 \sim \|\cdot\|_2 \iff \exists a, b > 0\colon \forall x \in X
  \implies a\|x_1\|_1 \leq \|x\|_2 \leq b \|x\|_1$
\end{stm}
\begin{thm}[Riesz]
  $X,\ \dim{X} < +\infty$ --- linear set. \\
  \underline{Then:} Any pair of norms in $X$ are equivalent.
\end{thm}
\begin{proof}
  $l_1, \dotsc, l_n$ --- linearly independent from $X$. $\forall x \in X =
  \sum\limits_{k = 1}^n \alpha_k l_K \\
  \bar{x} \leftrightarrow (l_1, \dotsc, l_n) = \bar{l} \in \R^n$ \\
  Let $\|\cdot\|$ --- some norm in $X$.
  \begin{align*}
  \|x\| \underset{\triangle}{\leq} \sum\limits_{k = 1}^n \|l_k\| |\alpha_k|
  \underset{\text{Cauchy}}{\leq} \underbrace{\sqrt{\sum\limits_{k=1}^n \|l_k\|^2}}_{const(B), B - basis}
  \equalto{\sqrt{\sum\limits_{k=1}^n |\alpha_k|^2}}{\|\bar{\alpha{}}\| = \|x\|_1} \\
  \|x\|_1 = \sqrt{\sum\limits_{k=1}^n \|\alpha_k\|^2},\ x = \sum \alpha_k l_k  \\
  \|x\| \leq b \|x\|_1 \\
  ?\exists a > 0\colon a\|x\|_1 \leq \|x\| \implies \|\cdot\| \sim \|\cdot\|_1 \\
  \end{align*}
  Let $f(\alpha_1, \dotsc, \alpha_n) = \norm{\sum\limits_{k = 1}^n \alpha_k l_k} \\
  \begin{aligned}[t]
    \\|f(\bar{\alpha} + \Delta\bar{\alpha}) - f(\bar\alpha)| = &\abs{\norm{\sum_{k=1}^n
    \alpha_k l_k + \sum\limits_{k=1}^n \Delta \alpha_k l_K } - \norm{\sum_{k=1}^n
    \alpha_k l_k}} \leq \\ & \norm{\sum_{k=1}^n \Delta \alpha_k l_k} \leq \tendsto{\sum
    \| l_k\| |\Delta \alpha_k|}{0, \Delta \alpha_k \to 0} \implies f
  \textnormal{--- continuous on}\ \R{}^n
  \end{aligned}$
  $S_1 = \{\sum\limits_{k=1}^n |\alpha_k|^2 = 1\} \subset \R{}^m$, f --- continuous
  on $S_1$, $S_1$ --- compact, $\bar{\alpha}^* \in S_1$ \\
  By Weierstrass theorem there exists a point $\alpha^* \in S_1$ on a sphere,
  in which function $f$ achieves its minimum
  $\implies \forall \alpha \in S_1\ f(\bar{\alpha}^*) \leq f(\bar{\alpha}) \\
  \textnormal{If}\ f(\bar{\alpha}^*) = 0,\
  \textnormal{then}\ \norm{\sum\limits_{k=1}^n \alpha_k^* l_k} = 0 \implies
  \sum\limits_{k=1}^n \alpha_k^* l_k = 0,\ \bar{\alpha}^* \in S_1, \\
  \textnormal{but}\ l_1 \dots l_n\ \textnormal{are linearly independent}\ \to\leftarrow
  \implies \min\limits_{S_1} f = m > 0 \\
  \begin{aligned}[t]
    \|x\| = \norm{\sum\limits_{k=1}^n \alpha_k l_k}  = f(\bar{\alpha}) =
    \sqrt{\sum\limits_{k=1}^n \alpha_k^2} \cdot \norm{\sum
    \underset{\beta_k}{\boxed{\frac{\alpha_k}{\sqrt{\sum_{k=1}^n \alpha_k^2}}}}
    \cdot l_k} & \geq ,\ \bar{\beta} = (\beta_1 \dots \beta_n) \in S_1 \\
    & \geq m \cdot \|x\|_1,\ a = m \qedhere 
  \end{aligned}$
\end{proof}
\begin{cor}
  $X$ --- NS, $Y \subset X, \dim{Y} < + \infty \implies Y = \cl{Y}$ \\
  In functional analysis linear subset (set of points, closed by scalar multiplication and
  addition) and linear subspace (closed linear subset) are different matters.
\end{cor}
\begin{proof}
  $Y = \alpha(l_1, \dotsc, l_n) = \Set{\sum\limits_{i=1}^n \alpha_il_i | l_1, \dotsc, l_n - \text{lin. indep.}} \\
  y_m \in Y, y_m \to y$ in $X \implies y \in Y?\\ 
  \norm{y_m - y} \to 0 \implies \norm{y_m - y_p} \to 0,\ m,p \to \infty \\
  \norm{y}, y \in Y.\\ 
  \textnormal{By Riesz theorem all norm in Y are equivalent.} \\
  y = \sum\limits_{j=1}^n \alpha_j l_J, \norm{y}_0 = \sqrt{\sum\limits_{j=1}^n
    \alpha_j^2}$ --- some norm (by linear independance). \\ 
  By Riesz theorem $\|y\| \sim \|y\|_0 \\
  \underbrace{\|y_m - y_p\|}_{\in Y} \to 0 \implies \|y_m - y_p\|_0 \to 0 \\
  \bar{\alpha}=(\alpha_1, \dotsc, l_n) \in \R^n\ y_m = \sum\limits_{i = 1}^n \alpha_i^{(m)}l_i \\
  |\alpha_i^{(m)} - \alpha_i^{(l)}| \to 0\ \forall i = 1, \dotsc, n;\
  \bar{\alpha} = (\alpha_1^{(m)}, \dotsc, \alpha_n^{(m)}) \to \alpha^* =
  (\alpha_1^*, \dotsc , \alpha_n^*) \\
  y^* = \sum\limits_{i=1}^n \alpha_i^* l_I \in Y,\ \|y_m - y\| \to 0, y = y^*
  \implies y \in Y \qedhere$
\end{proof}
\begin{defn}
  If normed space if complete, then it is called \textbf{B-space or Banach space}.
\end{defn}
\begin{ex}
 $C[a, b]$ --- functions continuous on $[a, b]$.
\end{ex}
\begin{ex}
  Lebesgue space, 
  $p \geq 1, L_p(E) = \Set{f - \textnormal{measurable on E},\ \int\limits_E \abs{f}^P < + \infty}$.
\end{ex}
If $X$ --- Banach space, \\
$\begin{aligned}[t]
& \sum_{n = 1}^\infty x_n = \lim_{n \to \infty}
\sum_{k = 1}^n x_k,\ \sum_{n=1}^\infty \|x_n\| < + \infty \\
& \|S_n - S_m\| = \norm{\sum_{k = m + 1}^n x_k} \leq \sum_{m + 1}^n
\|x_k\| \xrightarrow[n, m \to \infty]{} 0 \\
& \implies \|S_n - S_m\| \to 0 \implies  \exists \lim_{n \to \infty} S_n,
\sum_{k=1}^n x_k \text{--- converges.}
\end{aligned}$ \\
In Banach spaces works the theory of absolute convergence of numeric series.
\begin{lemma}[Riesz]
  
\end{lemma}

