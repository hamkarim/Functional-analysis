\section{Normed spaces}
\begin{defn}
  $X$ --- \textbf{linear set}, $x + y, \alpha \cdot x$, $\alpha \in \R$ \\
  The purpose of norm definition, is to construct a topology on $X$, so that 2 linear operations are continuous on it.
\end{defn}

$\phi\colon X \to \R\colon$
\begin{enumerate}
\item $\phi(x) \geq 0,\ = 0 \iff x = 0$
\item $\phi(\alpha x) = |\alpha| \phi(x)$
\item $\phi(x + y) \leq \phi(x) + \phi(y)$
\end{enumerate}

\begin{defn}
  $\phi$ --- \textbf{norm} on $X$, $\phi(x) = \|x\|$
\end{defn}

$\rho(x, y) \defeq \|x - y\|$ --- metric on $X$.

\begin{defn}
  $\\(X, \|\cdot\|)$ --- \textbf{normed space} --- special case of metrical space.
\end{defn}

\noindent
$x = \lim x_n \overset{def}{\implies} \rho(x_n, x) \to 0 \iff \|x_n - x\| \to 0$

\begin{stm}
  In the topology of a normed space linear operations are continuous on $X$.
\end{stm}

\begin{proof}\leavevmode
  \begin{enumerate}
  \item
    $\begin{aligned}[t]
      x_n \to x,\ y_n \to y;\ \|(x_n + y_n) - (x + y)\| & = \|(x_n - x) + (y_n - y)\|  \leq \\
      & \leq  \tendsto{\|x_n - x\|}{0} + \tendsto{\|y_n - y\|}{0} \\
      & \implies x_n + y_n \to x + y
    \end{aligned}$
  \item
    $\begin{aligned}[t]
       \alpha_n \to \alpha,\ x_n \to x;\ \|\alpha_n x_n - \alpha x\| & =
        \|(\alpha_n - \alpha)x_n + \alpha(x_n - x)\| \leq \\
        & \leq \tendsto{|\alpha_n - \alpha|}{0} \cdot \underbrace{\|x_n\|}_{bounded} + \tendsto{\alpha\|x_n - x\|}{0}
    \end{aligned}$ \\
    $x_n \to x \implies \|x_n\|$ --- bounded. \\
    $\alpha_x x_n \to \alpha x$ \qedhere
  \end{enumerate}
\end{proof}

\begin{stm}
  From the triangle inequality $\bigl|\|x\| - \|y\|\bigr| \leq \|x - y\| \\
  x_n \to x \implies \|x_n\| \to \|x\|$ \\
  Norm is continious.
\end{stm}

\begin{ex}
  $\R^n$
  \begin{enumerate}
  \item $\|\bar{x}\| = \sqrt{\sum\limits_{k = 1}^n x_k^2}$
  \item $\|\bar{x}\|_1 \defeq \sum\limits_{k = 1}^n|x_k|$
  \item $\|\bar{x}\|_2 \defeq \max \{|x_1|, \dotsc, |x_n|\}$
  \item $C[a, b]$ --- functions continuous on $[a, b];\ \|f\| = \max\limits_{x \in [a, b]}|f(x)|$
  \item $L_p (E) = \Set{f \textnormal{ --- measurable},\ \int\limits_E \abs{f}^p < +
    \infty}\\
    p \geq 1,\ \|f\|_p = \biggl(\displaystyle\int\limits_E |f|^p\biggr)^{\dfrac 1p}$
  \end{enumerate}
\end{ex}

Because the set of points is the same, arises the question about convergence
comparison. \\
$\|\cdot\|_1 \sim \|\cdot\|_2,\ x_n \overset{\|\cdot\|_1}{\to} x \iff x_n \overset{\|\cdot\|_2}{\to} x$

\begin{stm}
  $\\\|\cdot\|_1 \sim \|\cdot\|_2 \iff \exists a, b > 0\colon \forall x \in X
  \implies a\|x_1\|_1 \leq \|x\|_2 \leq b \|x\|_1$
\end{stm}

\begin{thm}[Riesz]
  $X,\ \dim{X} < +\infty$ --- linear set. \\
  \underline{Then:} Any pair of norms in $X$ are equivalent.
\end{thm}

\begin{proof}
  $l_1, \dotsc, l_n$ --- linearly independent from $X$. $\forall x \in X =
  \sum\limits_{k = 1}^n \alpha_k l_K$
  \[\bar{x} \leftrightarrow (l_1, \dotsc, l_n) = \bar{l} \in \R^n\]
  Let $\|\cdot\|$ --- some norm in $X$.
  \begin{gather*}
  \|x\| \underset{\triangle}{\leq} \sum\limits_{k = 1}^n \|l_k\| |\alpha_k|
  \underset{\text{Cauchy}}{\leq} \underbrace{\sqrt{\sum\limits_{k=1}^n \|l_k\|^2}}_{const(B), B - basis}
  \equalto{\sqrt{\sum\limits_{k=1}^n |\alpha_k|^2}}{\|\bar{\alpha{}}\| = \|x\|_1} \\
  \|x\|_1 = \sqrt{\sum\limits_{k=1}^n \|\alpha_k\|^2},\ x = \sum \alpha_k l_k  \\
  \|x\| \leq b \|x\|_1 \\
  ?\exists a > 0\colon a\|x\|_1 \leq \|x\| \implies \|\cdot\| \sim \|\cdot\|_1 \\
  \end{gather*}
  Let $f(\alpha_1, \dotsc, \alpha_n) = \norm{\sum\limits_{k = 1}^n \alpha_k l_k}$
  \begin{align*}
    \\|f(\bar{\alpha} + \Delta\bar{\alpha}) - f(\bar\alpha)| = &
      \abs{\rule{0em}{2em}\norm{\sum_{k=1}^n
    \alpha_k l_k + \sum_{k=1}^n \Delta \alpha_k l_K } - \norm{\sum_{k=1}^n
    \alpha_k l_k}} \leq \\ & \norm{\sum_{k=1}^n \Delta \alpha_k l_k} \leq \tendsto{\sum
    \| l_k\| |\Delta \alpha_k|}{0, \Delta \alpha_k \to 0} \implies \text{$f$ is continuous on $\R^n$}
  \end{align*}

  $S_1 = \Set{\sum\limits_{k=1}^n \abs{\alpha_k}^2 = 1} \subset \R^m$, f --- continuous
  on $S_1$, $S_1$ --- compact, $\bar{\alpha}^* \in S_1$ \\
  By Weierstrass theorem there exists a point $\alpha^* \in S_1$ on a sphere,
  in which function $f$ achieves its minimum
  $\implies \forall \alpha \in S_1\ f(\bar{\alpha}^*) \leq f(\bar{\alpha})$

  If $f(\bar{\alpha}^*) = 0$,
  then $\norm{\sum\limits_{k=1}^n \alpha_k^* l_k} = 0 \implies
  \sum\limits_{k=1}^n \alpha_k^* l_k = 0,\ \bar{\alpha}^* \in S_1$, \\
  but $l_1 \dots l_n$ are linearly independent $\contr
  \implies \min\limits_{S_1} f = m > 0$
  \begin{align*}
    \|x\| = \norm{\sum_{k=1}^n \alpha_k l_k}  = f(\bar{\alpha}) =
    \sqrt{\sum_{k=1}^n \alpha_k^2} \cdot \norm{\sum
    \underset{\beta_k}{\boxed{\dfrac{\alpha_k}{\sqrt{\sum_{k=1}^n \alpha_k^2}}}}
    \cdot l_k} & \geq ,\ \bar{\beta} = (\beta_1 \dots \beta_n) \in S_1 \\
    & \geq m \cdot \|x\|_1,\ a = m \qedhere
  \end{align*}
\end{proof}

\begin{cor}
  $X$ --- NS, $Y \subset X, \dim{Y} < + \infty \implies Y = \cl{Y}$
  \begin{note}
    Functional analysis differentiates between linear subset (set of points,
    closed by addition and scalar multiplication) and linear subspace (closed
    linear subset).
  \end{note}
\end{cor}

\begin{proof}
  $Y = \L(l_1, \dotsc, l_n) = \Set{\sum\limits_{i=1}^n \alpha_i l_i | l_1, \dotsc, l_n - \text{lin. indep.}} \\
  y_m \in Y, y_m \to y$ in $X \implies y \in Y?\\
  \norm{y_m - y} \to 0 \implies \norm{y_m - y_p} \to 0,\ m,p \to \infty \\
  \norm{y}, y \in Y.\\
  \textnormal{By Riesz theorem all norm in Y are equivalent.} \\
  y = \sum\limits_{j=1}^n \alpha_j l_J, \norm{y}_0 = \sqrt{\sum\limits_{j=1}^n
    \alpha_j^2}$ --- some norm (by linear independance). \\
  By Riesz theorem $\|y\| \sim \|y\|_0 \\
  \underbrace{\|y_m - y_p\|}_{\in Y} \to 0 \implies \|y_m - y_p\|_0 \to 0$ \\
  Notice that convergence by $\|\cdot\|_0$ is coordinatewise. \\
  $\bar{\alpha}=(\alpha_1, \dotsc, \alpha_n) \in \R^n\ y_m = \sum\limits_{i = 1}^n \alpha_i^{(m)}l_i \\
  |\alpha_i^{(m)} - \alpha_i^{(l)}| \to 0\ \forall i = 1, \dotsc, n;\
  \bar{\alpha} = (\alpha_1^{(m)}, \dotsc, \alpha_n^{(m)}) \to \alpha^* =
  (\alpha_1^*, \dotsc , \alpha_n^*)\\
  y^* = \sum\limits_{i=1}^n \alpha_i^* l_I \in Y,\ \|y_m - y\| \to 0$ \\
  Bu the limit uniqueness $y = y^* \implies y \in Y$
\end{proof}

\begin{defn}
  If normed space if complete, then it is called \textbf{B-space} or \textbf{Banach space}.
\end{defn}

\begin{ex}
 $C[a, b]$ --- functions continuous on $[a, b]$.
\end{ex}

\begin{ex}
  Lebesgue space,
  $p \geq 1, L_p(E) = \Set{f \textnormal{ is measurable on $E$},\ \int\limits_E \abs{f}^P < + \infty}$.
\end{ex}
If $X$ --- Banach space,
\begin{align*}
& \sum_{n = 1}^\infty x_n = \lim_{n \to \infty}
\sum_{k = 1}^n x_k,\ \sum_{n=1}^\infty \|x_n\| < + \infty \\
& \|S_n - S_m\| = \norm{\sum_{k = m + 1}^n x_k} \leq \sum_{m + 1}^n
\|x_k\| \xrightarrow[n, m \to \infty]{} 0 \\
& \implies \|S_n - S_m\| \to 0 \implies  \exists \lim_{n \to \infty} S_n,
\sum_{k=1}^n x_k \text{ converges.}
\end{align*}

In Banach spaces works the theory of absolute convergence of numerical series.

\begin{lemma}[Riesz's lemma about almost perpendicular]
  $Y$ --- eigen subspace of $X$ --- normed space.
  $\forall \epsilon \in (0, 1)\ \exists z_\epsilon \in X \colon$
\begin{enumerate}
\item $z_\epsilon \notin Y$
\item $\|z_\epsilon\| = 1$
\item $\rho(z_\epsilon, Y) > 1 - \epsilon$
\end{enumerate}
\end{lemma}

\begin{proof}
  $\exists x \in X \setminus Y,\ d = \rho(x, Y)$ \\
  Suppose $d = 0$ then, $\exists y_n \in Y\colon \|x - y_n\| < \dfrac{1}{n},\ n \to \infty,\ y_n \to x \\
  Y = \cl{Y} \implies x \in Y \contr x \notin Y,\ d > 0 \\
  \forall \epsilon \in (0, 1)\ \dfrac{1}{1 - \epsilon} > 1\ \exists y_\epsilon
  \in Y\colon \|x - y_\epsilon\| < \dfrac{1}{1 - \epsilon}d \\
  z_\epsilon = \dfrac{x - y_\epsilon}{\|x - y_\epsilon\|},\ \|z_\epsilon\| = 1 \\
  \forall y \in Y\ \|z_\epsilon - y\| = \norm{\dfrac{x - y_\epsilon}{\|x -
      y_\epsilon\|} - y} = \dfrac{\|x - \overbrace{(y_\epsilon + \|x - y_\epsilon\| \cdot y)}^{\in Y}\|
    \geq d}{\|x- y_\epsilon\| < \frac{1}{1 - \epsilon}d} > 1 - \epsilon$
\end{proof}

\begin{cor}
  $X$ --- normed space, $\dim{X} = +\infty,\ S = \Set{x | \|x\| = 1}$, then
  closed unit ball $\overline{B}$ is not compact in it.
\end{cor}

\begin{proof}
  $\forall x_1 \in S,\ Y_1 = \L\{x_1\}$ --- finite dimensional linear set.
  $\implies\ \text{closed in}\ X \implies Y_1$ --- subspace. \\
  $\dim{X} = + \infty > \dim{Y_1} \implies Y_1$ --- eigen subspace. \\
  Then by the Riesz lemma $(\epsilon = \frac 12)$:
  \begin{align*}
    & \exists x_2 \in X \colon \|x_2\| = 1,\ \|x_2 - x_1\| > \dfrac 12\ \textnormal{(Notice that $x_2$ appears to be an element of $S$)} \\
    & Y_2 = \L\{x_1, x_2\}\ \exists x_3 \in S\colon \|x_3 - x_j\| > \dfrac 12,\ j = 1,2
  \end{align*}
  Continue by induction. Because $\dim{X} = + \infty$ the process willl never
  finish. \\
  $x_n \in S \colon \|x_n - x_m\| > \dfrac 12,\ n \neq m$ --- obviously we
  cannot extract converging subsequence. $\implies S$ --- not a compact. And
  that means that $\overline{B} \supset S$ is not a compact either. 
\end{proof}
