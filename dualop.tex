\section{Dual operators}
\begin{defn}
  Let $X$ be normed space and $X^*$ it's dual space,\\
  $A \in L(X, Y)$, $\phi \in Y^*$\\ 
  $\forall x \in X\ f(x) = \phi(Ax) \implies f \in X^*$ \\
  $\phi \in Y^* \mapsto f = \phi \circ A \in X^*$ \\
  $A^* \colon Y^* \to X^*$ \\
  Then $A^*(\phi) = \phi \circ A$ is an operator \textbf{dual} to $A$.
\end{defn}

\begin{ex}[Hilbert space]
  \begin{gather*}
    \forall f \in H^* \\
    f(x, y) = \inprod{x, y},\ \|f\| = \|y\| \\
    A \colon H_1 \to H_2,\ A^* \colon H_2^* \to H_1^* \\
    \inprod{Ax, y} = \inprod{x, A^*y}
  \end{gather*}
\end{ex}

\begin{ex}\leavevmode
  \begin{enumerate}
  \item
    $\begin{aligned}[t]
      &A \colon R^n \to R^n \\
      &(a_{ij})\cdot
        \begin{pmatrix}
          x_1\\
          \vdots\\
          x_n
        \end{pmatrix} =
      \begin{pmatrix}
        y_1\\
        \vdots\\
        y_m
      \end{pmatrix}
    \end{aligned}$\\
    It is obvious that dual operator $A^*$ is just $A^T$.
  \item
    $\begin{aligned}[t]
      &l_2,\ (\vec{e_1}, \dotsc, \vec{e_n}) \text{ --- basis} \\
      &\vec{e}_n = (0, \dotsc, 0, 1, 0, \dotsc) \\
      &\overline{x} = \sum_{n = 1}^\infty x_n e_n,\ x_n = \inprod{\overline{x}, e_n} \\
      &\lambda_n : |\lambda_n| \leq M\ A \overline{x}  = \sum_{n = 1}^\infty \lambda_n x_n \overline{e}_n \in l_2 \\
      &A \colon l_2 \to l_2\ A^* y = \sum_{n = 1}^\infty \overline{\lambda}_n y_n e_n\\
      &\inprod{Ax, y} = \inprod{x, A^* y} 
    \end{aligned}$
  \end{enumerate}
\end{ex}

\begin{stm}
    $\|A^*\| = \|A\|$
\end{stm}

\begin{proof} 
  \begin{gather*}
    f = A^* \phi = \phi \circ A \\
    |f(x)| = |\phi (Ax)| \leq \|\phi\| \cdot \|Ax\| \leq \|\phi\| \cdot \|A\| \cdot \|x\| \\
    \|f\| \leq \|A\| \cdot \|\phi\| \\
    \|A^*\phi\| \leq \|A\| \cdot \|\phi\|\\
    \|A^*\| \leq \|A\|
  \end{gather*}
  Now let's check the inverse $\|A\| \leq \|A^*\|$?
  \begin{gather*}
    \|A\| = \sup_{\|x\| = 1} \|Ax\| = \sup_{\|x\| \leq 1} \|Ax\| \\
    \forall \epsilon > 0\ \exists x_\epsilon : \|x_\epsilon\| = 1 : \|A\| -
    \epsilon < \|A x_\epsilon\| \\
    Ax_\epsilon \in Y\ \text{By Hahn-Banach theorem:} \\
    \exists \phi : \|\phi\| = 1 : \phi(Ax_\epsilon) = \|Ax_\epsilon\| \\
    \|A\| - \epsilon < |\phi(Ax_\epsilon)| \leq \underbrace{\|\phi\|}_{1} \|Ax_\epsilon\| \\
    \|A\| - \epsilon < \|A x_\epsilon\| = \|A^* \phi(x_\epsilon)\| \leq \|A^*
    \phi\| \cdot \underbrace{\|x_\epsilon\|}_{1} = \|A^*\phi\| \leq \|A^*\| \cdot
    \underbrace{\|\phi\|}_{1} \\
    \|A\| - \epsilon \leq \|A^*\| \\
    \|A\| \leq \|A^*\| \qedhere
  \end{gather*}
\end{proof}

\begin{defn}
  $X, X^*, \forall S \subset X$ \\
  $S^\perp = \Set{f \in X^* | \forall x \in S \implies f(x) = 0}$ ---
  \textbf{orthogonal addition in dual space}. \\
  $S \subset X^*\ S^\perp = \Set{x \in X | \forall f \in S \implies f(x) = 0}$
  --- \textbf{orthogonal addition in $X$}.\\
  Both of theese orthogonal additions are closed sets.
\end{defn}

\begin{stm}\leavevmode
  \begin{enumerate}
  \item $0 \in X \implies X^* = \{0\}^\perp$
  \item $0 \in X^* \implies X = \{0\}^\perp$
  \end{enumerate}
\end{stm}

\begin{proof}\leavevmode
  \begin{enumerate}
  \item
    $\begin{aligned}[t]
      &X^* = \{0\}^\perp ? \\
      &f \in X^*\ f(0) = 0, f \in \{0\}^\perp
    \end{aligned}$
  \item
    $\begin{aligned}[t]
      &X = \{0\}^\perp ?\\
      &\forall x \in X\ 0(x) = 0 \\
      &X \in \{0\}^\perp
    \end{aligned}$ 
  \end{enumerate}
  In both cases the inverse relation is obvioius.
\end{proof}

\begin{stm}
  $X^\perp = {0}, {X^*}^\perp = {0}$
\end{stm}

\begin{proof}\leavevmode
  \begin{enumerate}
  \item 
    $\begin{aligned}[t]
      &0 \in X^*\ \forall x \in X 0(x) = 0 \\
      &0 \in X^\perp \\
      &f \in X^\perp\ \forall x \in X\ f(x) = 0 \implies f = 0 \\
      &X^\perp = \{0\} \\
    \end{aligned}$
  \item
    $\begin{aligned}[t]
      &0 \in X\ \forall f\ f(0) = 0\\
      &0 \in {X^*}^\perp \\
      &x \in {X^*}^\perp,\ x \neq 0\\
      &\exists f: \|f\| = 1 : f(x) = \|x\| \neq 0
    \end{aligned}$ \\
     ${X^*}^\perp = \{0\}$ \qedhere
  \end{enumerate}
\end{proof}

\noindent
\begin{gather*}
  A \colon X \to Y \\
  A x = y \\
  y \in R(A)?
\end{gather*}

The main diffuclty here, is that it is hard to describe the set of values of a
particular linear operator.
Using the orthogonal addition and dual operator definitions we can describe the
most common criteria of points being contained in operator's values set. At first we need to
make sure that this set is closed, i.e. $R(A) = \cl{R(A)}$.

\begin{thm}[1]
  \[
    \cl{R(A)} = (\ker{A^*})^\perp
  \]
\end{thm}

\begin{thm}[2]
  \[
    R(A) = \cl{R(A)} \implies R(A^*) = (\ker{A})^\perp
  \]
\end{thm}