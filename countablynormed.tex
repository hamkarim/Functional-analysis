\section{Countably-normed spaces.}
\begin{defn}
  $X$ --- linear set, $p$ --- \textbf{seminorm}:
  \begin{enumerate}
  \item $p(x) \geq 0$
  \item $p(\lambda x) = |\lambda| p(x)$
  \item $p(x + y) \geq p(x) + p(y)$ 
  \end{enumerate}
\end{defn}

\begin{defn}
  $p_1, p_2, \dotsc, p_n, \dotsc$ --- seminorms \\
  $\forall n\ p_n(x) = 0 \implies x
  = 0$ $(X, p_1, p_2, \dotsc, p_n, \dotsc )$ --- \textbf{countably normed space}.
\end{defn}

\noindent
$x = \lim x_m \iff \forall n \in \N \lim\limits_{m \to \infty} (x_m - x ) = 0$
with respect to corresponding seminorn $p_n$\\
If in countably-normed space we assume $\rho(x, y) = \sum\limits_{n = 1}^\infty
\dfrac{1}{2^n}\dfrac{p_n(x - y)}{1 + p_n(x - y)}$, we will get a valid metric.
Thus countably-normed space is always metrizable.
In countably-normed space two linear operations $(x + y, \lambda x)$ are
continuous, which means any countably normed space is also a topological vector
space.

\begin{ex}
  $C^\infty[a, b] = \Set{x(t), t \in [a, b] | x(t) \text{--- infinetely diff.}} \\
  p_n(x) = \max\limits_{[a, b]}\abs{x^{(n)}(t)}\ n = 0,1,2,\dotsc$
\end{ex}

From the next theorem we will see that $C^\infty[a, b]$ is non-normalizable (has no norm convergence
by which is equivalent to seminorm convergence).
We will also try to deduce the criterion of countably-normed space normalizability.

\begin{defn}
  System of seminorms is called \textbf{monotone} if $\forall x \in X,\
  \forall n \in \N\ p_n(x) \leq p_{n + 1}(x)$
\end{defn}

\begin{defn}
  $\{p_n\} \sim \{q_n\}$ if they have the same convergence (limits in both systems
  are equal).
\end{defn}

\begin{defn}
  $p_m \in \{p_n\}$ is called \textbf{essential} if it can not be majorized by any of
  the preceeding seminorms. $p_m$ is majorized by $p_n$ if $\exists C: \forall x \in X\
  p_m(x) \leq C \cdot p_n(x)$.
\end{defn}

\begin{stm}
  For system of seminorms there exists equivalent monotone system.
\end{stm}

\begin{proof}
  Let $q_n(x) = \sum\limits_{k = 1}^n p_k(x)$, it is obvious that every $q_n$ is
  seminorm. $\{q_n\} \sim \{p_n\}?$ \\
  $p_n(x_m - x) \to 0 \implies \sum\limits_{k = 1}^n p_k(x_m - x) \to 0 \implies
  q_n(x_m - x) \to 0$. Backwards proof is the same.
\end{proof}

\noindent
This statement allows us to operate only on monotone seminorm systems.
\begin{stm}
  \label{stm:seminorm_equiv}
  Two monotone seminorm systems are equivalent if and only if they majorize
  each other, i.e. for any seminorm $p_n$ from $\{p_n\}$ there exists majorizing
  seminorm $q_m$ from $\{q_m\}$ and vice verca.
\end{stm}

\begin{proof}
  If two systems majorize each other obviously they are equivalent. 
  Let two systems be equivalent. $\{p_n\} \sim \{q_n\}$ \\
  $\{p_n\},\ \{q_n\}\ \forall p_n\ \exists q_m \colon\ \exists C\ \forall
  x \in X\ p_n(x) \leq C \cdot q_m'(x)$ \\
  Proof by contradiction. Let there be some $p_{n_0} \colon$
  \begin{align*}
    &\forall q_m \exists x_m \in X \colon p_{n_0}(x_m) \geq m \cdot q_m(x_m) \\
    &q_m(\dfrac{x_m}{p_{n_0}(x_m)}) \leq \dfrac 1m,\ y_m =
      \dfrac{x_m}{p_{n_0}(x_m)},\ q_m(y_m) \leq \dfrac 1m,\ p_{n_0}(y_m) = 1
  \end{align*}

  Let's fix $m_0$, because $q_m$ is monotone:
  \begin{align*}
    &m \geq m_0,\ q_{m_0}(y_m) \leq q_m(y_m) \leq \dfrac{1}{m} \\ 
    &q_{m_0}(y_m) \leq \dfrac 1m,\ m \geq m_0,\ m \to +\infty \\
    &q_{m_0}(y_m) \to 0\quad \forall m_0\ q_m(y_m) \to 0
  \end{align*}

  But $p_{n_0}(y_m) = 1 \not\to 0,\ m\to \infty\ \contr$
\end{proof}

\begin{thm}[Normalizability criterion]
  Countably-normed space with monotone seminorm system is normalizable if and
  only if this system has finite number of essential seminorms.
\end{thm}

\begin{proof}
  $\leftarrow$ \\
  Let system has finite amount of essential seminorms. 
  $\{p_{n_1}, \dotsc p_{n_m}\}$ \\
  $\|x\| = \sum\limits_{k = 1}^{n_m}p_k(x)$ --- norm. Given system and this
  majorize each other. \\
  $\Rightarrow$ \\
  Easily proovable using statement \ref{stm:seminorm_equiv}
\end{proof}

\begin{ex}
  $\R^\infty\ \bar{x} = (x_1, \dotsc, x_n, \dotsc)\ p_n(\bar{x}) = |x_n|$. All
  seminorms are essential, thus $\R^\infty$ is not normalizable. The same
  applies to $C^\infty[a, b]$.
\end{ex}
